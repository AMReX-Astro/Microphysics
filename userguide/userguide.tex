\documentclass[11pt]{book} 

\tolerance=600

% for \begin{center}, etc.
\usepackage[margin=1.0in]{geometry}

% all kinds of math macros
\usepackage{amsmath}
\usepackage{amssymb}

% eps figures
\usepackage{epsfig}

% chapter title styles
\usepackage[Sonny]{fncychap}
\ChNameVar{\LARGE}
\ChTitleVar{\LARGE\sl}

% part page style see
% http://tex.stackexchange.com/questions/6609/problems-with-part-labels-using-titlesec
\usepackage{titlesec}

\titleformat{\part}[display]
   {\Huge\filcenter}
   {{\partname{}} \thepart}
   {0em}
   {\hrule}


% hyperlinks -- load after fncychap
\usepackage{hyperref}

% color package
\usepackage[usenames]{color}

% make the MarginPars look pretty
\setlength{\marginparwidth}{0.75in}
\newcommand{\MarginPar}[1]{\marginpar{\vskip-\baselineskip\raggedright\tiny\sffamily
\hrule\smallskip{\color{red}#1}\par\smallskip\hrule}}

% to increase the likelihood that floats will occur "here" when you
% want them to
\renewcommand{\floatpagefraction}{1.0}
\renewcommand{\topfraction}{1.0}
\renewcommand{\bottomfraction}{1.0}
\renewcommand{\textfraction}{0.0}

% number subsubsections and put them in the TOC
\setcounter{tocdepth}{3}
\setcounter{secnumdepth}{3}

% custom hrule for title page
\newcommand{\HRule}{\rule{\linewidth}{0.125mm}}


% short table of contents
\usepackage{shorttoc}

% spacing in the table of contents
\usepackage[titles]{tocloft}

\setlength{\cftbeforechapskip}{2ex}
\setlength{\cftbeforesecskip}{0.25ex}

% For splitting up lists into multitple columns
\usepackage{multicol}

% don't put a header on blank pages, see
% http://www.latex-community.org/forum/viewtopic.php?f=4&p=51559
% change ``plain'' to ``empty'' to eliminate the page number
\makeatletter
\renewcommand*\cleardoublepage{\clearpage\if@twoside
\ifodd\c@page\else
\hbox{}
\thispagestyle{empty}
\newpage
\if@twocolumn\hbox{}\newpage\fi\fi\fi}
\makeatother


% don't make the chapter/section headings uppercase.  See the fancyhdr
% documentation (section 9)
\usepackage{fancyhdr}
\renewcommand{\chaptermark}[1]{%
 \markboth{\chaptername
\ \thechapter.\ #1}{}}

\renewcommand{\sectionmark}[1]{\markright{\thesection---#1}}

\graphicspath{{CastroVerification/}{ConvertCheckpoint/}{Parallel/}{Scaling/}{Visualization/}}

% skip a bit of space between paragraphs, to enhance readability
\usepackage{parskip}



% special fraction
\newcommand{\sfrac}[2]{\mathchoice
  {\kern0em\raise.5ex\hbox{\the\scriptfont0 #1}\kern-.15em/
   \kern-.15em\lower.25ex\hbox{\the\scriptfont0 #2}}
  {\kern0em\raise.5ex\hbox{\the\scriptfont0 #1}\kern-.15em/
   \kern-.15em\lower.25ex\hbox{\the\scriptfont0 #2}}
  {\kern0em\raise.5ex\hbox{\the\scriptscriptfont0 #1}\kern-.2em/
   \kern-.15em\lower.25ex\hbox{\the\scriptscriptfont0 #2}}
  {#1\!/#2}}

\def\Ab {{\bf A}}
\def\eb {{\bf e}}
\def\Fb {{\bf F}}
\def\gb {{\bf g}}
\def\Hb {{\bf H}}
\def\ib {{\bf i}}
\def\Ib {{\bf I}}
\def\Kb {{\bf K}}
\def\lb {{\bf l}}
\def\Lb {{\bf L}}
\def\nb {{\bf n}}
\def\Pb {{\bf P}}
\def\Qb {{\bf Q}}
\def\rb {{\bf r}}
\def\Rb {{\bf R}}
\def\Sb {{\bf S}}
\def\ub {{\bf u}}
\def\Ub {{\bf U}}
\def\xb {{\bf x}}

\def\dt       {\Delta t}
\def\omegadot {\dot\omega}

\def\inp  {{\rm in}}
\def\outp {{\rm out}}
\def\sync {{\rm sync}}

\def\half   {\frac{1}{2}}
\def\myhalf {\sfrac{1}{2}}
\def\nph    {{n+\myhalf}}

% codes
\newcommand{\castro}{{\sf Castro}}
\newcommand{\maestro}{{\sf Maestro}}
\newcommand{\boxlib}{{\sf BoxLib}}
\newcommand{\microphysics}{{\sf Microphysics}}
\newcommand{\yt}{{\sf yt}}
\newcommand{\amrvis}{{\sf Amrvis}}
\newcommand{\visit}{{\sf VisIt}}

\usepackage{listings}

\definecolor{gray}{rgb}       {0.8,0.8,0.8}
\definecolor{light-blue}{rgb} {0.8,0.8,1.0}
\definecolor{light-green}{rgb}{0.8,1.0,0.8}
\definecolor{light-red}{rgb}  {1.0,0.9,0.9}

\lstset{
  basicstyle=\small\ttfamily,%
  frame=shadowbox,%
  rulesepcolor=\color{gray},%
  backgroundcolor=\color{white}%
}


%------------------------------------------------------------------------------
\begin{document}

\frontmatter

\begin{titlepage}
\begin{center}
\ \\[3in]
{\sf \Huge Microphysics} 

\begin{minipage}{5.5in}
\HRule\\[2mm]
\centering
{\Large \em A collection of astrophysical microphysics routines \\ with interfaces to the BoxLib codes}

\HRule
\end{minipage}

\ \\[1 in]
{\sf \huge User's Guide}

\vfill

{\large \today}
\end{center}

\end{titlepage}


\shorttoc{Chapter Listing}{0}

\setcounter{tocdepth}{2}
\tableofcontents

\clearpage

\chapter*{Preface}
\chaptermark{Preface}
\addcontentsline{toc}{chapter}{preface}

Welcome to the \microphysics\ User's Guide!

In this User's Guide we describe the microphysics modules designed to support \boxlib\ 
codes such as \castro\ and \maestro. These all have a consistent interface and 
are designed to provide the users of those codes an easy experience in moving from 
the barebones microphysics modules provided in those codes. For the purposes of this 
user's guide, the microphysical components we currently deal with are the equation 
of state (EOS) and the nuclear burning network.

\microphysics\ is not a stand-alone code. It is intended to be used in conjuction with 
\boxlib\ codes and so we do not provide support for running these codes separately. 
However, in many cases we will provide test modules that demonstrate a minimal working 
example for how to run the modules.

\clearpage

\mainmatter


\chapter{The Basics}

Getting started with \microphysics is straightforward. Because the modules here are 
already in a format that the \boxlib\ codes understand, you only need to 
provide to the code calling these routines their location on your system. The 
code will do the rest. To do so, define the {\tt MICROPHYSICS\_DIR} 
environment variable, either at a command line or (if you use the {\tt bash} terminal) 
through your {\tt $\sim$/.bashrc} profile, e.g.:
\begin{equation*}
  \texttt{export MICROPHYSICS\_DIR = /path/to/Microphysics}
\end{equation*}
The calling code will know that when you provide it an {\tt EOS\_dir} and {\tt Network\_dir},
that it should look in this repository and not in its own microphysics location.

\chapter{Data Structures}
\label{sec:data_structures}

All of the routines in this software package are standardized so that you interact with them 
using the same type of data structure. We call this the {\tt eos\_t\_vector}. This is a 
Fortran 90 derived data type (which, for our purposes, is analogous to a C++ struct). It is 
a collection of data specifying the microphysical state of the fluid that we are evaluating. 
This derived type has many components, and in the Fortran syntax we access them with the 
\% operator. For a particular instantation named {\tt eos\_state}, the most important data is 
the following. Most of these components are one-dimensional (rank-1) Fortran arrays that 
consist of pointers to the corresponding data, except for a few with additional information.
\begin{itemize}
  \item {\tt eos\_state \% rho}: density.
  \item {\tt eos\_state \% T}: temperature.
  \item {\tt eos\_state \% p}: pressure.
  \item {\tt eos\_state \% e}: internal energy.
  \item {\tt eos\_state \% h}: enthalpy.
  \item {\tt eos\_state \% s}: entropy.
  \item {\tt eos\_state \% xn}: mass fractions of species (this is a 2D array, where the second 
                                index has the same number of components as there are fluid species)
  \item {\tt eos\_state \& aux}: any auxiliary variables carried with the fluid (also a 2D array, where
                                 the second index has as many components as there are auxiliary variables)
\end{itemize}
There is a lot more information that can be saved here, such as the partial derivatives of the 
thermodynamic state variables with respect to each other. To see a complete list, examine the 
{\tt eos\_type.f90} file inside the code calling \microphysics (e.g. {\tt Castro/EOS/eos\_type.f90}.)

\chapter{Equations of State}

In this chapter on equations of state, we list the EOS routines available for your use, and 
then we describe the correct structure of an EOS module in case you want to build your own.

\section{EOS List}

Here we list the equations of state provided in this repository.

\begin{itemize}

\item {\tt gamma\_law\_general} represents a gamma law gas, with equation of state:
\begin{equation}
  P = (\gamma - 1) \rho e.
\end{equation}
The gas is currently assumed to be monatomic and ideal. The entropy comes from the Sackur-Tetrode equation.

\item {\tt polytrope} represents a polytropic fluid, with equation of state:
\begin{equation}
  P = K \rho^\gamma.
\end{equation}
The gas is also assumed to obey the above gamma law equation of state
connecting the pressure and internal energy. Therefore $\rho$ is the
only independent variable; there is no temperature dependence. The
user either selects from a set of predefined options reflecting
physical polytropes (e.g. a non-relativistic, fully degenerate
electron gas) or inputs their own values for $K$ and $\gamma$.

\item {\tt ztwd} is the zero-temperature degenerate electron equation of state 
of Chandrasekhar (1935), which is designed to describe white dward material. The 
pressure satisfies the equation:
\begin{equation}
  P(x) = A \left( x(2x^2-3)(x^2 + 1)^{1/2} + 3\, \text{sinh}^{-1}(x) \right),
\end{equation}
with $A = \pi m_e^4 c^5 / (3 h^3)$. Here $x$ is a dimensionless measure of density,
with $\rho = B x^3$ and $B = 8\pi \mu_e m_p m_e^3 c^3 / (3h^3)$, 
where $\mu_e$ is the electron fraction and $h$ is the Planck constant. The enthalpy 
was worked out by Hachisu (1986):
\begin{equation}
  h(x) = \frac{8A}{B}\left(x^2 + 1\right)^{1/2}.
\end{equation}
The internal energy satisfies the standard relationship to the enthalpy:
\begin{equation}
  e = h - p / \rho.
\end{equation}
Since the pressure-density relationship does not admit a closed-form 
solution for the density in terms of the pressure, if we call the EOS 
with pressure as a primary input then we do Newton-Raphson iteration 
to find the density that matches this pressure.

\item {\tt multigamma} is an ideal gas equation of state where each
  species can have a different value of $\gamma$.  This mainly affects
  how the internal energy is constructed as each species, represented
  with a mass fraction $X_k$ will have its contribution to the total
  specific internal energy take the form of $e = p/\rho/(\gamma_k -
  1)$.  The main thermodynamic quantities take the form:
\begin{align}
p &= \frac{\rho k T}{m_u} \sum_k \frac{X_k}{A_k} \\
e &= \frac{k T}{m_u} \sum_k \frac{1}{\gamma_k - 1} \frac{X_k}{A_k} \\
h &= \frac{k T}{m_u} \sum_k \frac{\gamma_k}{\gamma_k - 1} \frac{X_k}{A_k}
\end{align}
We recognize that the usual astrophysical $\bar{A}^{-1} = \sum_k
X_k/A_k$, but now we have two other sums that involve different
$\gamma_k$ weightings.

The specific heats are constructed as usual, 
\begin{align}
c_v &= \left . \frac{\partial e}{\partial T} \right |_\rho = 
    \frac{k}{m_u} \sum_k \frac{1}{\gamma_k - 1} \frac{X_k}{A_k} \\
c_p &= \left . \frac{\partial h}{\partial T} \right |_p = 
    \frac{k}{m_u} \sum_k \frac{\gamma_k}{\gamma_k - 1} \frac{X_k}{A_k} 
\end{align}
and it can be seen that the specific gas constant, $R \equiv c_p - c_v$ is
independent of the $\gamma_i$, and is simply $R = k/m_u\bar{A}$ giving the
usual relation that $p = R\rho T$.  Furthermore, we can show
\begin{equation}
\Gamma_1 \equiv \left . \frac{\partial \log p}{\partial \log \rho} \right |_s =  
   \left ( \sum_k \frac{\gamma_k}{\gamma_k - 1} \frac{X_k}{A_k} \right ) \bigg /
   \left ( \sum_k \frac{1}{\gamma_k - 1} \frac{X_k}{A_k} \right ) =
\frac{c_p}{c_v} \equiv \gamma_\mathrm{effective} 
\end{equation}
and $p = \rho e (\gamma_\mathrm{effective} - 1)$.

This equation of state takes several runtime parameters that can set the
$\gamma_i$ for a specific species.  These are set in the {\tt \&extern}
namelist in the {\tt probin} file.  The parameters are:
\begin{itemize}
\item {\tt eos\_gamma\_default}: the default $\gamma$ to apply for
  all species
\item {\tt species\_X\_name} and {\tt species\_X\_gamma}: set the $\gamma_i$
  for the species whose name is given as {\tt species\_X\_name} to the
  value provided by {\tt species\_X\_gamma}.  Here, {\tt X} can be one
  of the letters: {\tt a}, {\tt b}, or {\tt c}, allowing us to specify
  custom $\gamma_i$ for up to three different species.
\end{itemize}

\item {\tt helmholtz} contains a general, publicly available
stellar equation of state based on the Helmholtz free energy,
with contributions from ions, radiation, and electron degeneracy, as
described in \cite{timmes:1999,timmes:2000,flash}.

We have modified this EOS a bit to fit within the context of our codes. The 
vectorization is explicitly thread-safe for use with OpenMP and OpenACC.
In addition, we have added the ability to perform a Newton-Raphson iteration 
so that if we call the EOS with density and energy (say), then we will 
iterate over temperature until we find the temperature that matches 
this density--energy combination. If we cannot find an appropriate temperature,
we will reset it to {\tt small\_temp}, which needs to be set in the 
equation of state wrapper module in the code calling this. However, 
there is a choice of whether to update the energy to match this 
temperature, respecting thermodynamic consistency, or to leave 
the energy alone, respecting energy conservation. This is controlled through the 
\texttt{eos\_input\_is\_constant} parameter in your \texttt{extern}
namelist in your probin file.

We thank Frank Timmes for permitting us to modify his code and publicly release 
it in this repository.

\item {\tt stellarcollapse} is the equation of state module provided on 
\href{stellarcollapse.org}{stellarcollapse.org}. It is designed to be used for 
core-collapse supernovae and is compatible with a large number of equations of 
state that are designed to describe matter near nuclear density. You will need to 
download an appropriate interpolation table from that site to use this.

\end{itemize}



\section{EOS Structure}

Each EOS should have two main routines by which it interfaces to the
rest of \castro.  At the beginning of the simulation, {\tt specific\_eos\_init}
will perform any initialization steps and save EOS variables (mainly
\texttt{smallt}, the temperature floor, and \texttt{smalld}, the
density floor). These variables are stored in the main EOS module of the 
code calling these routines. This would be the appropriate time for, say, 
loading an interpolation table into memory.

The main evaluation routine is called {\tt specific\_eos}. It should accept 
an {\tt eos\_input} and an {\tt eos\_state}; see Section \ref{sec:data_structures}.



\chapter{Nuclear Networks}

In this chapter on nuclear networks, we list the network routines available for your use, and 
then we describe the correct structure of a network module in case you want to build your own.

\section{Network List}

\begin{itemize}

\item {\tt general\_null} directory is a bare interface for a
  nuclear reaction network; no reactions are enabled, and no
  auxiliary variables are accepted. It contains several sets of
  isotopes; for example,
  {\tt Networks/general\_null/triple\_alpha\_plus\_o.net} would describe the
  triple-$\alpha$ reaction converting helium into carbon, as well as
  oxygen and iron.

\item {\tt Networks/ignition\_simple} directory contains a single-step
  $^{12}\mathrm{C}(^{12}\mathrm{C},\gamma)^{24}\mathrm{Mg}$ reaction.
  The carbon mass fraction equation appears as
\begin{equation}
\frac{D X(^{12}\mathrm{C})}{Dt} = - \frac{1}{12} \rho X(^{12}\mathrm{C})^2
    f_\mathrm{Coul} \left [N_A \left <\sigma v \right > \right]\enskip,
\end{equation}
where $N_A \left <\sigma v\right>$ is evaluated using the reaction
rate from (Caughlan and Fowler 1988).  The Coulomb screening factor,
$f_\mathrm{Coul}$, is evaluated using the general routine from the
Kepler stellar evolution code (Weaver 1978), which implements the work
of (Graboske 1973) for weak screening and the work of (Alastuey 1978
and Itoh 1979) for strong screening.

\end{itemize}

\section{Network Structure}

There are two primary files within each network directory, each with associated
data-storing modules. The first pair, {\tt actual\_network.f90} and
{\tt actual\_network\_data.f90}, supply the number and names of species and
auxiliary variables, as well as other initializing data, 
such as their mass numbers, proton numbers, and binding energies. It needs to 
define the {\tt nspec} and {\tt naux} quantities as integer parameters.

The second pair, {\tt actual\_burner.f90} and {\tt actual\_burner\_data.f90},
contains the burner routine, which takes an {\tt eos\_t} as an input, and
returns another as an output. It also takes the current simulation and time,
and the integration timestep. It needs to define the {\tt nrates} quantity as
an integer parameter.

The integration equations are:
\begin{eqnarray}
\frac{dX_k}{dt} &=& \omegadot_k(\rho,X_k,T)\enskip, \label{eq:VODE1C} \\
\frac{dT}{dt} &=&\frac{1}{c_p} \left ( -\sum_k \xi_k  \omegadot_k  \right )\enskip. \label{eq:tempreactC}
\end{eqnarray}
We integrate these using the stiff ordinary differential equation integration methods
provided by the VODE package.  The absolute error tolerances are set
to $10^{-12}$ for the species, and a relative tolerance of $10^{-6}$
is used for the temperature.  The integration yields the new values of
the mass fractions, $X_k^\outp$.  Equation (\ref{eq:tempreactC}) is
derived from equation (???) by assuming that the pressure is constant
during the burn state.  In evolving these equations, we need to
evaluate $c_p$ and $\xi_k$.  In theory, this means evaluating the
equation of state for each right-hand side evaluation that VODE
requires.  In practice, we freeze $c_p$ and $\xi_k$ at the start of
the integration time step and compute them using $\rho^\inp,
X_k^\inp,$ and $T^\inp$ as inputs to the equation of state.  Note that
the density remains unchanged during the burning.  At the end of the
routine, we compute $T^\outp = T(\rho^\outp,e^{\outp},X_k^\outp)$.

%------------------------------------------------------------------------------
\backmatter

\renewcommand\bibname{References}
\addcontentsline{toc}{chapter}{References}
\bibliographystyle{plain}
\bibliography{refs}

\end{document}
