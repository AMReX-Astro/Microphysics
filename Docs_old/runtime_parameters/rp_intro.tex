\label{chapter:parameters}

The behavior of the network and EOS are controlled by many runtime
parameters.  These parameters are defined in plain-text files {\tt
  \_parameters} located in the different directories that hold the
microphysics code.  At compile time, a script in the \amrex\ bulid
system, {\tt findparams.py}, locates all of the {\tt \_parameters}
files that are needed for the given choice of network, integrator, and
EOS, and assembles all of the runtime parameters into a module named
{\tt extern\_probin\_module} (using the {\tt write\_probin.py}
script).  

Note: depending on the application code, the filename of the source 
file that contains the {\tt extern\_probin\_module} may differ
(in \castro\ it is {\tt extern.f90}, and uses the {\tt \&extern} namelist;
in \maestro\ it is in the main {\tt probin.f90} and uses the same namelist
as general \maestro\ runtime parameters).

Parameter definitions take the form of:
\begin{verbatim}
# comment describing the parameter
name              data-type       default-value      priority
\end{verbatim}
Here, the {\tt priority} is simply an integer.  When two directories
define the same parameter, but with different defaults, the version of
the parameter with the highest priority takes precedence.  This allows
specific implementations to override the general parameter defaults.

The documentation below is automatically generated, using the comments
in the {\tt \_parameters} files.
