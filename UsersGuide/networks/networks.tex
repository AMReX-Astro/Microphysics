
In this chapter on nuclear networks, we list the network routines
available for your use, and then we describe the correct structure of
a network module in case you want to build your own.

\section{Network List}

\begin{itemize}

\item {\tt general\_null} directory is a bare interface for a nuclear
  reaction network; no reactions are enabled, and no auxiliary
  variables are accepted. It contains several sets of isotopes; for
  example, {\tt Networks/general\_null/triple\_alpha\_plus\_o.net}
  would describe the triple-$\alpha$ reaction converting helium into
  carbon, as well as oxygen and iron.

\item {\tt Networks/ignition\_simple} directory contains a single-step
  $^{12}\mathrm{C}(^{12}\mathrm{C},\gamma)^{24}\mathrm{Mg}$ reaction.
  The carbon mass fraction equation appears as
\begin{equation}
\frac{D X(^{12}\mathrm{C})}{Dt} = - \frac{1}{12} \rho X(^{12}\mathrm{C})^2
    f_\mathrm{Coul} \left [N_A \left <\sigma v \right > \right]\enskip,
\end{equation}
where $N_A \left <\sigma v\right>$ is evaluated using the reaction
rate from (Caughlan and Fowler 1988).  The Coulomb screening factor,
$f_\mathrm{Coul}$, is evaluated using the general routine from the
Kepler stellar evolution code (Weaver 1978), which implements the work
of (Graboske 1973) for weak screening and the work of (Alastuey 1978
and Itoh 1979) for strong screening.

\end{itemize}

\section{Network Structure}

There are two primary files within each network directory, each with
associated data-storing modules. The first pair, {\tt
  actual\_network.f90} and {\tt actual\_network\_data.f90}, supply the
number and names of species and auxiliary variables, as well as other
initializing data, such as their mass numbers, proton numbers, and
binding energies. It needs to define the {\tt nspec} and {\tt naux}
quantities as integer parameters.

The second pair, {\tt actual\_burner.f90} and {\tt
  actual\_burner\_data.f90}, contains the burner routine, which takes
an {\tt eos\_t} as an input, and returns another as an output. It also
takes the current simulation and time, and the integration
timestep. It needs to define the {\tt nrates} quantity as an integer
parameter.

The integration equations are:
\begin{eqnarray}
\frac{dX_k}{dt} &=& \omegadot_k(\rho,X_k,T)\enskip, \label{eq:VODE1C} \\
\frac{dT}{dt} &=&\frac{1}{c_p} \left ( -\sum_k \xi_k  \omegadot_k  \right )\enskip. \label{eq:tempreactC}
\end{eqnarray}
We integrate these using the stiff ordinary differential equation
integration methods provided by the VODE package.  The absolute error
tolerances are set to $10^{-12}$ for the species, and a relative
tolerance of $10^{-6}$ is used for the temperature.  The integration
yields the new values of the mass fractions, $X_k^\outp$.  Equation
(\ref{eq:tempreactC}) is derived from equation (???) by assuming that
the pressure is constant during the burn state.  In evolving these
equations, we need to evaluate $c_p$ and $\xi_k$.  In theory, this
means evaluating the equation of state for each right-hand side
evaluation that VODE requires.  In practice, we freeze $c_p$ and
$\xi_k$ at the start of the integration time step and compute them
using $\rho^\inp, X_k^\inp,$ and $T^\inp$ as inputs to the equation of
state.  Note that the density remains unchanged during the burning.
At the end of the routine, we compute $T^\outp =
T(\rho^\outp,e^{\outp},X_k^\outp)$.
