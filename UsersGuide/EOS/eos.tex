
In this chapter on equations of state, we list the EOS routines
available for your use, and then we describe the correct structure of
an EOS module in case you want to build your own.

\section{EOS List}

Here we list the equations of state provided in this repository.

\begin{itemize}

\item {\tt gamma\_law\_general} represents a gamma law gas, with
  equation of state:
\begin{equation}
  P = (\gamma - 1) \rho e.
\end{equation}
The gas is currently assumed to be monatomic and ideal. The entropy
comes from the Sackur-Tetrode equation.

\item {\tt polytrope} represents a polytropic fluid, with equation of
  state:
\begin{equation}
  P = K \rho^\gamma.
\end{equation}
The gas is also assumed to obey the above gamma law equation of state
connecting the pressure and internal energy. Therefore $\rho$ is the
only independent variable; there is no temperature dependence. The
user either selects from a set of predefined options reflecting
physical polytropes (e.g. a non-relativistic, fully degenerate
electron gas) or inputs their own values for $K$ and $\gamma$.

\item {\tt ztwd} is the zero-temperature degenerate electron equation
  of state of Chandrasekhar (1935), which is designed to describe
  white dward material. The pressure satisfies the equation:
\begin{equation}
  P(x) = A \left( x(2x^2-3)(x^2 + 1)^{1/2} + 3\, \text{sinh}^{-1}(x) \right),
\end{equation}
with $A = \pi m_e^4 c^5 / (3 h^3)$. Here $x$ is a dimensionless
measure of density, with $\rho = B x^3$ and $B = 8\pi \mu_e m_p m_e^3
c^3 / (3h^3)$, where $\mu_e$ is the electron fraction and $h$ is the
Planck constant. The enthalpy was worked out by Hachisu (1986):
\begin{equation}
  h(x) = \frac{8A}{B}\left(x^2 + 1\right)^{1/2}.
\end{equation}
The internal energy satisfies the standard relationship to the enthalpy:
\begin{equation}
  e = h - p / \rho.
\end{equation}
Since the pressure-density relationship does not admit a closed-form
solution for the density in terms of the pressure, if we call the EOS
with pressure as a primary input then we do Newton-Raphson iteration
to find the density that matches this pressure.

\item {\tt multigamma} is an ideal gas equation of state where each
  species can have a different value of $\gamma$.  This mainly affects
  how the internal energy is constructed as each species, represented
  with a mass fraction $X_k$ will have its contribution to the total
  specific internal energy take the form of $e = p/\rho/(\gamma_k -
  1)$.  The main thermodynamic quantities take the form:
\begin{align}
p &= \frac{\rho k T}{m_u} \sum_k \frac{X_k}{A_k} \\
e &= \frac{k T}{m_u} \sum_k \frac{1}{\gamma_k - 1} \frac{X_k}{A_k} \\
h &= \frac{k T}{m_u} \sum_k \frac{\gamma_k}{\gamma_k - 1} \frac{X_k}{A_k}
\end{align}
We recognize that the usual astrophysical $\bar{A}^{-1} = \sum_k
X_k/A_k$, but now we have two other sums that involve different
$\gamma_k$ weightings.

The specific heats are constructed as usual,
\begin{align}
c_v &= \left . \frac{\partial e}{\partial T} \right |_\rho = 
    \frac{k}{m_u} \sum_k \frac{1}{\gamma_k - 1} \frac{X_k}{A_k} \\
c_p &= \left . \frac{\partial h}{\partial T} \right |_p = 
    \frac{k}{m_u} \sum_k \frac{\gamma_k}{\gamma_k - 1} \frac{X_k}{A_k} 
\end{align}
and it can be seen that the specific gas constant, $R \equiv c_p -
c_v$ is independent of the $\gamma_i$, and is simply $R =
k/m_u\bar{A}$ giving the usual relation that $p = R\rho T$.
Furthermore, we can show
\begin{equation}
\Gamma_1 \equiv \left . \frac{\partial \log p}{\partial \log \rho} \right |_s =  
   \left ( \sum_k \frac{\gamma_k}{\gamma_k - 1} \frac{X_k}{A_k} \right ) \bigg /
   \left ( \sum_k \frac{1}{\gamma_k - 1} \frac{X_k}{A_k} \right ) =
\frac{c_p}{c_v} \equiv \gamma_\mathrm{effective} 
\end{equation}
and $p = \rho e (\gamma_\mathrm{effective} - 1)$.

This equation of state takes several runtime parameters that can set
the $\gamma_i$ for a specific species.  These are set in the {\tt
  \&extern} namelist in the {\tt probin} file.  The parameters are:
\begin{itemize}
\item {\tt eos\_gamma\_default}: the default $\gamma$ to apply for all
  species
\item {\tt species\_X\_name} and {\tt species\_X\_gamma}: set the
  $\gamma_i$ for the species whose name is given as {\tt
  species\_X\_name} to the value provided by {\tt species\_X\_gamma}.
  Here, {\tt X} can be one of the letters: {\tt a}, {\tt b}, or {\tt
    c}, allowing us to specify custom $\gamma_i$ for up to three
  different species.
\end{itemize}

\item {\tt helmholtz} contains a general, publicly available stellar
  equation of state based on the Helmholtz free energy, with
  contributions from ions, radiation, and electron degeneracy, as
  described in \cite{timmes:1999,timmes:2000,flash}.

We have modified this EOS a bit to fit within the context of our
codes. The vectorization is explicitly thread-safe for use with OpenMP
and OpenACC.  In addition, we have added the ability to perform a
Newton-Raphson iteration so that if we call the EOS with density and
energy (say), then we will iterate over temperature until we find the
temperature that matches this density--energy combination. If we
cannot find an appropriate temperature, we will reset it to {\tt
  small\_temp}, which needs to be set in the equation of state wrapper
module in the code calling this. However, there is a choice of whether
to update the energy to match this temperature, respecting
thermodynamic consistency, or to leave the energy alone, respecting
energy conservation. This is controlled through the
\texttt{eos\_input\_is\_constant} parameter in your \texttt{extern}
namelist in your probin file.

We thank Frank Timmes for permitting us to modify his code and
publicly release it in this repository.

\item {\tt stellarcollapse} is the equation of state module provided
  on \href{stellarcollapse.org}{stellarcollapse.org}. It is designed
  to be used for core-collapse supernovae and is compatible with a
  large number of equations of state that are designed to describe
  matter near nuclear density. You will need to download an
  appropriate interpolation table from that site to use this.

\end{itemize}



\section{EOS Structure}

Each EOS should have two main routines by which it interfaces to the
rest of \castro.  At the beginning of the simulation, {\tt
  specific\_eos\_init} will perform any initialization steps and save
EOS variables (mainly \texttt{smallt}, the temperature floor, and
\texttt{smalld}, the density floor). These variables are stored in the
main EOS module of the code calling these routines. This would be the
appropriate time for, say, loading an interpolation table into memory.

The main evaluation routine is called {\tt specific\_eos}. It should
accept an {\tt eos\_input} and an {\tt eos\_state}; see Section
\ref{sec:data_structures}.



