\label{ch:unit_tests}

\section{Comprehensive Unit Tests}

There are 2 unit tests in \microphysics\ that operate on a generic EOS
or reaction network.  To allow these to compile independent of any
application code (e.g., \maestro\ or \castro), copies of the EOS
driver {\tt eos.f90} and network interface {\tt network.f90} are
provided in {\tt Microphysics/unit\_test/}.  These tests compile using
the \amrex\ F90 build system.  The file {\tt
  Microphysics/GMicrophysics.mak} provides the macros necessary to
build the executable.  Runtime parameters are parsed in the same
fashion as in the application codes, using the {\tt write\_probin.py}
script.

\subsection{EOS test}

{\tt Microphysics/unit\_test/test\_eos/} is a unit test for the
equations of state in \microphysics.  It sets up a cube of data, with
$\rho$, $T$, and $X_k$ varying along the three dimensions, and then
calls the EOS in each zone.  Calls are done to exercise all modes of
calling the EOS, in order:

\begin{itemize}
  \item {\tt eos\_input\_rt}: We call the EOS using $\rho, T$.  this
    is the reference call, and we save the $h$, $e$, $p$, and $s$ from
    here to use in subsequent calls.

  \item {\tt eos\_input\_rh}: We call the EOS using $\rho, h$, to
    recover the original $T$.  To give the root finder some work to
    do, we perturb the initial temperature.  

    We store the relative error in $T$ in the output file.

  \item {\tt eos\_input\_tp}: We call the EOS using $T, p$, to
    recover the original $\rho$.  To give the root finder some work to
    do, we perturb the initial density.

    We store the relative error in $\rho$ in the output file.

  \item {\tt eos\_input\_rp}: We call the EOS using $\rho, p$, to
    recover the original $T$.  To give the root finder some work to
    do, we perturb the initial temperature.

    We store the relative error in $T$ in the output file.

  \item {\tt eos\_input\_re}: We call the EOS using $\rho, e$, to
    recover the original $T$.  To give the root finder some work to
    do, we perturb the initial temperature.

    We store the relative error in $T$ in the output file.

  \item {\tt eos\_input\_ps}: We call the EOS using $p, s$, to
    recover the original $\rho, T$.  To give the root finder some work to
    do, we perturb the initial density and temperature.

    Note: entropy is not well-defined for some EOSs, so we only attempt
    the root find if $s > 0$. 

    We store the relative error in $\rho, T$ in the output file.

  \item {\tt eos\_input\_ph}: We call the EOS using $p, h$, to
    recover the original $\rho, T$.  To give the root finder some work to
    do, we perturb the initial density and temperature.

    We store the relative error in $\rho, T$ in the output file.

  \item {\tt eos\_input\_th}: We call the EOS using $T, h$, to
    recover the original $\rho$.  To give the root finder some work to
    do, we perturb the initial density.

    Note: for some EOSs, $h = h(\rho)$ (e.g., and ideal gas), so there
    is no temperature dependence, and we do not do this test.

    We store the relative error in $\rho$ in the output file.

\end{itemize}

This unit test is marked up with OpenMP directives and therefore also
tests whether the EOS is threadsafe.

To compile for a specific EOS, e.g., {\tt helmholtz}, do:
\begin{verbatim}
make EOS_DIR=helmholtz -j 4
\end{verbatim}

Examining the output (a \amrex\ plotfile) will show you how big the
errors are.  You can use the {\tt AMReX/Tools/Postprocessing/} too
{\tt fextrema} to display the maximum error for each variable.

Note: an analogous {\tt test\_eos} exists in \maestro.

\subsection{Network test}

{\tt Microphysics/unit\_test/test\_react/} is a unit test for the
nuclear reaction networks in \microphysics.  It sets up a cube of
data, with $\rho$, $T$, and $X_k$ varying along the three dimensions
(as a $16^3$ domain), and then calls the EOS in each zone.  

The state in each zone of our data cube is determined by the runtime
parameters {\tt dens\_min}, {\tt dens\_max}, {\tt temp\_min}, and {\tt
  temp\_max} for $(\rho, T)$.  Because each network carries different
compositions, we specify explicitly the mass fraction of each species
for every step in the composition dimension in a file {\tt xin.*} for
each network.  Note: it is up to the user to ensure that they species
are in the proper order in that file and sum to 1.  The name of the
file to use is specified by the runtime parameter {\tt xin\_file}.

This test calls the network on each zone, running for a time {\tt
  tmax}.  The full state, including new mass fractions and energy
release is output to a \amrex\ plotfile.

You can compile for a specific integrator (e.g., {\tt BS}) or 
network (e.g., {\tt aprox13}) as:
\begin{verbatim}
make NETWORK_DIR=aprox13 INTEGRATOR_DIR=BS -j 4
\end{verbatim}

The loop over the burner is marked up for both OpenMP and OpenACC and
therefore this test can be used to assess threadsafety of the burners
as well as to optimize the GPU performance of the burners.



\section{Individual Network Tests}

Many of the networks have a subdirectory {\tt test/} (e.g. {\tt
  Microphysics/networks/triple\_alpha\_plus\_cago/test/}).  There are
usually 3 different drivers there that can be used to evaluate the
network or Jacobian on a single state:
\begin{itemize}
  \item {\tt eval.f90}

  \item {\tt testburn.f90}

  \item {\tt testjacobian.f90}
\end{itemize}
These all use the F90 \amrex\ build system and the macros in 
{\tt GMicrophysics.mak} to build the executable.  To make 
individual tests you can use the {\tt programs} variable, e.g.,
\begin{verbatim}
make programs=eval
\end{verbatim}
