\section{Introduction}

SDC provides a means to more strongly couple the reactions to the
hydrodynamics by evolving the reactions together with an approximation
of the advection over the timestep.

We want to solve the coupled equations:
\begin{equation}
\Uc_t = \Advs{\Uc} + \Rb(\Uc)
\end{equation}
where $\Uc$ is the conserved state vector, $\Uc = (\rho, (\rho X_k),
(\rho \Ub), (\rho E))^\intercal$, $X_k$ are the mass fractions
constrained via $\sum_k X_k = 1$, $\Ub$ is the velocity vector, and
$E$ is the specific total energy, related to the specific internal
energy, $e$, via $E = e + |\Ub|^2/2$.  Here $\Advs{\Uc}$ is the
advective source term (including any hydrodynamical sources),
\begin{equation}
\Advs{\Uc} = - \nabla \cdot \Fb(\Uc) + \Sb_\mathrm{hydro}
\end{equation}
and $\Rb(\Uc)$
is the reaction source term.


\section{Interface and Data Structures}

\subsection{\tt sdc\_t}

To accommodate the increased information required to evolve the
coupled system, we introduce a new data type, {\tt sdc\_t}.  This is
used to pass information to/from the integration routine from the
hydrodynamics code.


\section{ODE system}

The reactions don't modify the total density, $\rho$, or momentum,
$\rho \Ub$, so our ODE system is just:
\begin{equation}
\frac{d}{dt}\left ( 
   \begin{array}{c} \rho X_k \\ \rho E \\  \rho e \end{array} 
\right ) = 
\left ( \begin{array}{c}
   \Adv{\rho X_k}^{n+1/2} \\ \Adv{\rho E}^{n+1/2} \\ \Adv{\rho e}^{n+1/2} \\
\end{array} \right ) +
\left (
   \begin{array}{c} \rho \omegadot_k \\ \rho \Sdot \\ \rho \Sdot \end{array}
\right )
\end{equation}
where we include $e$ in addition to $E$ to provide additional thermodynamic 
information to help find a consistent $T$.  Here the advective courses
are piecewise-constant approximations to the change in the state due
to the hydrodynamics, computed with the during the hydro step.

However, to define the temperature, we need the kinetic energy (and
hence momentum and density) at any intermediate time, $t$.  We construct
these as needed from 


\section{Interfaces}

\subsection{{\tt actual\_integrator}}

SDC integration provides its own implementation of the main entry
point, {\tt actual\_integrator}.
\begin{lstlisting}[language={[95]fortran}]
  subroutine actual_integrator(state_in, state_out, dt, time)

    type (sdc_t), intent(in   ) :: state_in
    type (sdc_t), intent(inout) :: state_out
    real(dp_t),    intent(in   ) :: dt, time
\end{lstlisting}
The main difference here is that the type that comes in and out of the
interface is no longer a {\tt burn\_t}, but instead is an {\tt
  sdc\_t}.

The flow of this main routine is simpler than the non-SDC version:
\begin{enumerate}
\item Convert from the {\tt sdc\_t} type to the integrator's internal
  representation (e.g., {\tt sdc\_to\_bs} converts from a {\tt bs\_t}
  for the {\tt BS} integrator).

  This copies the state variables and advective sources into the
  integration type.  Since we only actually integrate $(\rho X_k),
  (\rho e), (\rho E)$, the terms corresponding to density and momentum
  are carried in an auxillary array (indexed through the {\tt rpar}
  mechanism).

\item Call the main integration routine to advance the inputs state
  through the desired time interval, producing the new, output state.

\item Convert back from the internal representation (e.g., a {\tt
  bs\_t}) to the {\tt sdc\_t} type.

\end{enumerate}

\subsection{Righthand side wrapper}

The manipulation of the righthand side is considerably more complex
now.  Each network only provides the change in molar
fractions \MarginPar{note: molar fractions} and internal energy, but
we need to convert these to the conservative system we are
integrating, including the advective terms.

\begin{enumerate}

\item Convert to a {\tt burn\_t} type, for instance via {\tt bs\_to\_burn}:

  \begin{enumerate}

  \item call {\tt fill\_unevolved\_variables} to update the density
    and momentum.  Since these don't depend on reactions, this is a
    simple algebraic update based on the piecewise-constant-in-time
    advective term:
    \begin{align}
      \rho(t) &= \rho^n + t \cdot \left [ \mathcal{A}(\rho) \right]^{n+1/2} \\
      (\rho \Ub)(t) &= (\rho \Ub)^n + t \cdot \left [ \mathcal{A}(\rho\Ub) \right]^{n+1/2} 
    \end{align}
    where here we assume that we are integrating in the ODE system
    starting with $t=0$.

  \item compute the specific internal energy, $e$, from either the
    total minus kinetic energy or internal energy carried by the
    integrator (depending on the value of {\tt T\_from\_eden}).

  \item get the temperature from the equation of state

  \item convert to a {\tt burn\_t} type, for instance via {\tt eos\_to\_burn}:

  \end{enumerate}

\item Call the network's {\tt actual\_rhs} routine to get just the 
 reaction sources to the update.  In particular, this returns
 the change in molar fractions, $\dot{Y}_k$ and the nuclear energy
 release, $\dot{S}$.

\item Convert back to the integrator's internal representation (e.g.,
  a {\tt bs\_t}, via {\tt rhs\_to\_bs})

  \begin{enumerate}
  \item call {\tt fill\_unevolved\_variables} \MarginPar{is this redundant?}

  \item fill the {\tt ydot} array in the integrator type (e.g., {\tt
    bs\_t}) with the advective sources that originally came into the
    intergrator through the {\tt sdc\_t}.

  \item Add the reacting terms.  This is done as:
    \begin{align}
      \dot{y}_{\rho X_k} &= \Adv{\rho X_k}^{n+1/2} + \rho A_k \dot{Y}_k \\
      \dot{y}_{\rho e} &= \Adv{\rho e}^{n+1/2} +\rho \dot{S} \\
      \dot{y}_{\rho E} &= \Adv{\rho E}^{n+1/2} + \rho \dot{S}
    \end{align}
      
  \end{enumerate}

\end{enumerate}

