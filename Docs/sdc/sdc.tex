\section{Introduction}

SDC provides a means to more strongly couple the reactions to the
hydrodynamics by evolving the reactions together with an approximation
of the advection over the timestep.

We want to solve the coupled equations:
\begin{equation}
\Uc_t + \Advs{\Uc} = \Rb(\Uc)
\end{equation}
where $\Uc$ is the conserved state vector, $\Uc = (\rho, (\rho X_k),
(\rho \Ub), (\rho E))^\intercal$, $X_k$ are the mass fractions
constrained via $\sum_k X_k = 1$, $\Ub$ is the velocity vector, and
$E$ is the specific total energy, related to the specific internal
energy, $e$, via $E = e + |\Ub|^2/2$.  Here $\Advs{\Uc}$ is the
advective term (including any hydrodynamical sources) and $\Rb(\Uc)$
is the reaction source term.


\section{Interface and Data Structures}

\subsection{\tt sdc\_t}

To accommodate the increased information required to evolve the
coupled system, we introduce a new data type, {\tt sdc\_t}.  This is
used to pass information to/from the integration routine from the
hydrodynamics code.


\section{ODE system}

The reactions don't modify the total density, $\rho$, or momentum,
$\rho \Ub$, so our ODE system is just:
\begin{equation}
\frac{d}{dt}\left ( 
   \begin{array}{c} \rho X_k \\ \rho E \\  \rho e \end{array} 
\right ) = 
\left ( \begin{array}{c}
   -\Adv{\rho X_k}^{n+1/2} \\ -\Adv{\rho E}^{n+1/2} \\ -\Adv{\rho e}^{n+1/2} \\
\end{array} \right ) +
\left (
   \begin{array}{c} \rho \omegadot_k \\ \rho \Sdot \\ \rho \Sdot \end{array}
\right )
\end{equation}
where we include $e$ in addition to $E$ to provide additional thermodynamic 
information to help find a consistent $T$.  Here the advective courses
are piecewise-constant approximations to the change in the state due
to the hydrodynamics, computed with the during the hydro step.

However, to define the temperature, we need the kinetic energy (and
hence momentum and density) at any intermediate time, $t$.  We construct
these as needed from 
