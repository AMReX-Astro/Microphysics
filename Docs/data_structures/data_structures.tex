\label{sec:data_structures}

All of the routines in this software package are standardized so that
you interact with them using the same type of data structure, a
Fortran derived type (which, for our purposes, is analogous to a
\cpp\ struct).

\section{EOS}

The main data structure for interacting with the EOS is {\tt eos\_t}.
This is a collection of data specifying the microphysical state of the
fluid that we are evaluating.  This derived type has many components,
and in the Fortran syntax we access them with the {\tt \%} operator. For a
particular instantation named {\tt eos\_state}, the most important
data is the following:
\begin{itemize}
  \item {\tt eos\_state \% rho}: density [$\mathrm{g~cm^{-3}}$]

  \item {\tt eos\_state \% T}: temperature [K]

  \item {\tt eos\_state \% p}: pressure [$\mathrm{erg~cm^{-3}}$]

  \item {\tt eos\_state \% e}: specific internal energy [$\mathrm{erg~g^{-1}}$]

  \item {\tt eos\_state \% h}: specific enthalpy [$\mathrm{erg~g^{-1}}$]

  \item {\tt eos\_state \% s}: specific entropy [$\mathrm{erg~g^{-1}~K^{-1}}$]

  \item {\tt eos\_state \% xn(:)}: mass fractions of species (this is an
    array, dimensioned to be the number of species, {\tt nspec})

  \item {\tt eos\_state \% aux(:)}: any auxiliary variables carried with
    the fluid (this is an array, dimensioned to be the number of 
    auxillary quantities, {\tt naux})
\end{itemize}
Note that both {\tt nspec} and {\tt naux} are meant to be properties of the
network, and they will come in through the {\tt network} module.

There is a lot more information that can be saved here, such as the
partial derivatives of the thermodynamic state variables with respect
to each other. To see a complete list, examine the {\tt eos\_type.f90}
file inside the code calling \microphysics\ (e.g. {\tt
  Castro/EOS/eos\_type.f90}).


\section{Networks}

The main data structure for interacting with the reaction networks is
{\tt burn\_t}.  This holds the composition (mass fractions),
thermodynamic state, and a lot of internal information used by the
reaction network (e.g. the righthand side of the ODEs, the Jacobian,
etc.).  Typically the user will only need to fill/use the following
information:
\begin{itemize}
\item {\tt burn\_state \% rho}: density [$\mathrm{g~cm^{-3}}$]

\item {\tt burn\_state \% T}: temperature [K]

\item {\tt burn\_state \% e}: the specific energy release from the
  burn---this is not the actual thermodynamic energy [$\mathrm{erg~g^{-1}}$] \MarginPar{clarify}

\item {\tt burn\_state \% h}: not used? \MarginPar{h?}

\item {\tt burn\_state \% xn(:)}: the mass fractions

\item {\tt burn\_state \% aux(:)}: any auxillary quantities (like $Y_e$)

\item {\tt burn\_state \% i}, {\% j}, {\% k}: hydrodynamic zone i, j, k for
  bug reporting, diagnostics

\item {\tt burn\_state \% time}: current simulation time

\item {\tt burn\_state \% shock}: are we inside of a hydrodynamic shock (the burner
  may wish to take action in this case)

\end{itemize}






\section{Converting Between Types}

There is significant overlap between {\tt eos\_t} and {\tt burn\_t}.
The {\tt burn\_type\_module} provides two routines, {\tt
  burn\_to\_eos} and {\tt eos\_to\_burn} that convert a {\tt burn\_t}
state to an {\tt eos\_t} state, and back.  Only the thermodynamic
variables that are common in the two types are copied.  This is
useful, for example, if you have a {\tt burn\_t} state and what to get
thermodynamic information by calling the EOS.


\section{Initial Values and Threadsafety}

