\documentclass[11pt]{book} 

\tolerance=600

% for \begin{center}, etc.
\usepackage[margin=1.0in]{geometry}

% longtable package used to split tables across pages                           
\usepackage{longtable}                                                          
                                                                                
% PDF-aware landscape package, used for rotating tables (including the          
% longtable)                                                                    
\usepackage{pdflscape}
\usepackage{rotating}

\input maestrosymbols

% all kinds of math macros
\usepackage{amsmath}
\usepackage{amssymb}

% eps figures
\usepackage{epsfig}

\usepackage{graphicx}

% chapter title styles
\usepackage[Sonny]{fncychap}
\ChNameVar{\LARGE}
\ChTitleVar{\LARGE\sl}

% part page style see
% http://tex.stackexchange.com/questions/6609/problems-with-part-labels-using-titlesec
\usepackage{titlesec}

\titleformat{\part}[display]
   {\Huge\filcenter}
   {{\partname{}} \thepart}
   {0em}
   {\hrule}


% hyperlinks -- load after fncychap
\usepackage{hyperref}

% color package
\usepackage[usenames]{color}

% table coloring                                                                
\usepackage{colortbl}                                                           
\definecolor{tableShade}{rgb}{0.945,0.961,0.980}                                
                                                                                
 

% make the MarginPars look pretty
\setlength{\marginparwidth}{0.75in}
\newcommand{\MarginPar}[1]{\marginpar{\vskip-\baselineskip\raggedright\tiny\sffamily
\hrule\smallskip{\color{red}#1}\par\smallskip\hrule}}

% to increase the likelihood that floats will occur "here" when you
% want them to
\renewcommand{\floatpagefraction}{1.0}
\renewcommand{\topfraction}{1.0}
\renewcommand{\bottomfraction}{1.0}
\renewcommand{\textfraction}{0.0}

% number subsubsections and put them in the TOC
\setcounter{tocdepth}{3}
\setcounter{secnumdepth}{3}

% custom hrule for title page
\newcommand{\HRule}{\rule{\linewidth}{0.125mm}}


% short table of contents
\usepackage{shorttoc}

% spacing in the table of contents
\usepackage[titles]{tocloft}

\setlength{\cftbeforechapskip}{2ex}
\setlength{\cftbeforesecskip}{0.25ex}

% For splitting up lists into multitple columns
\usepackage{multicol}

% don't put a header on blank pages, see
% http://www.latex-community.org/forum/viewtopic.php?f=4&p=51559
% change ``plain'' to ``empty'' to eliminate the page number
\makeatletter
\renewcommand*\cleardoublepage{\clearpage\if@twoside
\ifodd\c@page\else
\hbox{}
\thispagestyle{empty}
\newpage
\if@twocolumn\hbox{}\newpage\fi\fi\fi}
\makeatother


% don't make the chapter/section headings uppercase.  See the fancyhdr
% documentation (section 9)
\usepackage{fancyhdr}
\renewcommand{\chaptermark}[1]{%
 \markboth{\chaptername
\ \thechapter.\ #1}{}}

\renewcommand{\sectionmark}[1]{\markright{\thesection---#1}}

\graphicspath{{networks/}}

% skip a bit of space between paragraphs, to enhance readability
\usepackage{parskip}




\def\Ab {{\bf A}}
\def\eb {{\bf e}}
\def\Fb {{\bf F}}
\def\gb {{\bf g}}
\def\Hb {{\bf H}}
\def\ib {{\bf i}}
\def\Ib {{\bf I}}
\def\Kb {{\bf K}}
\def\lb {{\bf l}}
\def\Lb {{\bf L}}
\def\nb {{\bf n}}
\def\Pb {{\bf P}}
\def\Qb {{\bf Q}}
\def\rb {{\bf r}}
\def\Rb {{\bf R}}
\def\Sb {{\bf S}}
\def\ub {{\bf u}}
\def\Ub {{\bf U}}
\def\xb {{\bf x}}

\def\dt       {\Delta t}
\def\omegadot {\dot\omega}

\def\inp  {{\rm in}}
\def\outp {{\rm out}}
\def\sync {{\rm sync}}

\def\half   {\frac{1}{2}}
\def\myhalf {\sfrac{1}{2}}
\def\nph    {{n+\myhalf}}

% codes
\usepackage{listings}

\definecolor{gray}{rgb}       {0.8,0.8,0.8}
\definecolor{light-blue}{rgb} {0.8,0.8,1.0}
\definecolor{light-green}{rgb}{0.8,1.0,0.8}
\definecolor{light-red}{rgb}  {1.0,0.9,0.9}

\lstset{
  basicstyle=\small\ttfamily,%
  frame=shadowbox,%
  rulesepcolor=\color{gray},%
  backgroundcolor=\color{white}%
}


%------------------------------------------------------------------------------
\begin{document}

\frontmatter

\begin{titlepage}
\begin{center}
\ \\[3in]
{\sf \Huge Microphysics} 

\begin{minipage}{5.5in}
\HRule\\[2mm]
\centering
{\Large \em A collection of astrophysical microphysics routines \\ with interfaces to the BoxLib codes}

\HRule
\end{minipage}

\ \\[1 in]
{\sf \huge User's Guide}

\vfill

{\large \today}
\end{center}

\end{titlepage}


\shorttoc{Chapter Listing}{0}

\setcounter{tocdepth}{2}
\tableofcontents

\clearpage

\chapter*{Preface}
\chaptermark{Preface}
\addcontentsline{toc}{chapter}{preface}

Welcome to the \microphysics\ User's Guide!

In this User's Guide we describe the microphysics modules designed to
support \boxlib\ codes such as \castro\ and \maestro. These all have a
consistent interface and are designed to provide the users of those
codes an easy experience in moving from the barebones microphysics
modules provided in those codes. For the purposes of this user's
guide, the microphysical components we currently deal with are the
equation of state (EOS) and the nuclear burning network.

\microphysics\ is not a stand-alone code. It is intended to be used in
conjuction with \boxlib\ codes and so we do not provide support for
running these codes separately.  However, in many cases we will
provide test modules that demonstrate a minimal working example for
how to run the modules.

\clearpage

\mainmatter


\chapter{The Basics}

\input basics/basics


\chapter{Data Structures}

\input data_structures/data_structures

\chapter{Runtime Parameters}

\input runtime_parameters/rp_intro.tex

\input runtime_parameters/runtime_parameters.tex


\chapter{Equations of State}
\label{chapter:eos}

\input EOS/eos.tex


\chapter{Nuclear Networks}
\label{chapter:networks}

\input networks/networks.tex

\chapter{Nuclear Burners}
\label{chapter:burners}

\input burners/burners.tex


\chapter{Unit Tests}

\input unit_tests/unit_tests.tex

%------------------------------------------------------------------------------
\backmatter

\renewcommand\bibname{References}
\addcontentsline{toc}{chapter}{References}
\bibliographystyle{plain}
\bibliography{refs}

\end{document}
