\section{Getting Started}

Getting started with \microphysics\ is straightforward. Because the
modules here are already in a format that the \boxlib\ codes
understand, you only need to provide to the code calling these
routines their location on your system. The code will do the rest. To
do so, define the {\tt MICROPHYSICS\_HOME} environment variable, either
at a command line or (if you use the {\tt bash} shell) through your
{\tt $\sim$/.bashrc} profile, e.g.:
\begin{equation*}
  \texttt{export MICROPHYSICS\_HOME = /path/to/Microphysics}
\end{equation*}

The calling code will know that when you provide it an {\tt
  EOS\_dir} \MarginPar{this is Castro specific} and {\tt
  Network\_dir}, that it should look in this repository and not in its
own microphysics location.


\section{Structure}

The high-level directory structure delineates the types of microphysics
and the generic solvers:

\begin{itemize}
\item {\tt Docs/}: this User's Guide

\item {\tt EOS/}: the various equations of state

\item {\tt integration/}: the ODE integration routines used for the
  reaction networks

\item {\tt interfaces/}: copies of the main derived types that are used to
  interface with the EOS and reaction networks.  Note that most application
  codes will have their own local copies.  These are provided for unit testing
  in \microphysics.

\item {\tt networks/}: the nuclear reaction networks.  This is mostly just the
  righthand side of the network, as the actual integrators are decoupled from
  the network.

\item {\tt neutrinos/}: neutino loss source terms for the network energy equation.

\item {\tt NSE/}: routines for approximating nuclear statistical equilibrium.

\item {\tt rates/}: common nuclear reaction rate modules used by some of the 
  networks.

\item {\tt screening/}: common electron screening factors used by some of the 
  reaction networks.

\item {\tt unit\_test/}: self-contained unit tests for \microphysics.  These don't
  need any application code to build, but will require \boxlib.

\item {\tt util/}: linear algebra solvers and other routines.

\end{itemize}



\section{Design Philosophy}

Any application that uses \microphysics\ will at minimum need to
choose an EOS and a network.  These two components work together.  The
design philosophy is that the EOS depends on the network, but not the
other way around.  The decision was made for the network to act as the
core module, and lots of code depends on it.  This avoids circular
dependencies by having the {\tt eos\_t} and {\tt burn\_t} be built on
top of the network.

The network is meant to store the properties of the species (typically
nuclear isotopes) including their atomic weights and numbers, and also
describes any links between the species when burning.  

The equation of state relates the thermodynamic properties of the
material.  It depends on the composition of the material, typically
specified via mass fractions of the species, and uses the properties
of the species defined by the network to interpret the state.

We try to maximize code reuse in the \microphysics\ source, so the
solvers (ODE integration for the network and Newton-Raphson root
finding for the EOS) is separated from the specific implmentations of
the microphysics.
