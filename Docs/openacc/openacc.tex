\section{General Ideas}



\section{Common Compiler Errors}

\subsection{PGI}

To get more detailed about what kernels are launching, do:
\begin{verbatim}
export PGI_ACC_NOTIFY=1
\end{verbatim}

\paragraph{Compilation errors}

\begin{itemize}

\item {\em Multiple exit statements}

\begin{verbatim}
PGF90-S-0155-Compiler failed to translate accelerator region (see -Minfo messages): 
Unexpected flow graph (../../integration/BS/stiff_ode.F90: 1)
single_step_rosen:
\end{verbatim}
This results when you have multiple {\tt exit} statements from a {\tt
  do}-loop.  You need to consolidate any error/convergence checking in
a loop and have at most a single {\tt exit} statement.

Note: sometimes a {\tt do while} can help here, but there is a sense
that {\tt while}-loops do not perform optimally on GPUs.

\item {\em ACC routines not in Fortran modules}

\begin{verbatim}
PGF90-S-0155-Procedures called in a compute region must have acc routine information: 
  dgefa (../../integration/BS/stiff_ode.F90: 711)
PGF90-S-0155-Accelerator region ignored; see -Minfo messages  
  (../../integration/BS/stiff_ode.F90)
\end{verbatim}

This occurs when a subroutine relies on another routine that is not part
of a Fortran 90 module.  In this case, even if that routine already has
\begin{verbatim}
!$acc routine seq
\end{verbatim}
we need to mark the {\em calling} routine as well, with:
\begin{verbatim}
!$acc routine(dgesl) seq
\end{verbatim}
(e.g., for the Fortran routine {\tt dgesl}).
\end{itemize}


\paragraph{Runtime errors}

\begin{itemize}

\item {\em Multi-d array copies}
\begin{verbatim}
Unhandled builtin: 601 (pgf90_mzero8)
PGF90-F-0000-Internal compiler error. Unhandled builtin function.       
  0 (../../networks/triple_alpha_plus_cago/actual_rhs.f90: 146)
PGF90/x86-64 Linux 16.5-0: compilation aborted
\end{verbatim}

This error results from doing a multi-d array copy (with Fortran
notation) in GPU code.  The fix is to explicitly write out a loop over
rows.

\item {\em Illegal memory access}

\begin{verbatim}
call to cuMemcpyDtoHAsync returned error 700: Illegal address during kernel execution
call to cuMemFreeHost returned error 700: Illegal address during kernel execution
\end{verbatim}
This indicates that you went out of bounds in memory access or,
sometimes it seems, generated some NaNs.

\end{itemize}


\section{Debugging}

\subsection{{\tt cuda-gdb}}

Basic debugging can be done using {\tt cuda-gdb}.  This will work just
like {\tt gdb} and can give you the name of a routine where a crash
occurred, but generally doesn't produce line numbers.
