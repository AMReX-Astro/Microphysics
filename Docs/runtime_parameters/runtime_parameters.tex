
\label{ch:parameters}

\begin{landscape}


{\small

\renewcommand{\arraystretch}{1.5}
%
\begin{center}
\begin{longtable}{|l|p{5.25in}|l|}
\caption[BS parameters.]{BS parameters.} \label{table: BS runtime} \\
%
\hline \multicolumn{1}{|c|}{\textbf{parameter}} & 
       \multicolumn{1}{ c|}{\textbf{description}} & 
       \multicolumn{1}{ c|}{\textbf{default value}} \\ \hline 
\endfirsthead

\multicolumn{3}{c}%
{{\tablename\ \thetable{}---continued}} \\
\hline \multicolumn{1}{|c|}{\textbf{parameter}} & 
       \multicolumn{1}{ c|}{\textbf{description}} & 
       \multicolumn{1}{ c|}{\textbf{default value}} \\ \hline 
\endhead

\multicolumn{3}{|r|}{{\em continued on next page}} \\ \hline
\endfoot

\hline 
\endlastfoot


\rowcolor{tableShade}
\verb=  ode_max_steps  = &   Maximum number of steps to use in the ODE integration  &  10000 \\
\verb=  ode_scale_floor  = &   Floor to use for the ODE scaling vector  &  1.d-6 \\


\end{longtable}
\end{center}

} % ends \small


{\small

\renewcommand{\arraystretch}{1.5}
%
\begin{center}
\begin{longtable}{|l|p{5.25in}|l|}
\caption[integration parameters.]{integration parameters.} \label{table: integration runtime} \\
%
\hline \multicolumn{1}{|c|}{\textbf{parameter}} & 
       \multicolumn{1}{ c|}{\textbf{description}} & 
       \multicolumn{1}{ c|}{\textbf{default value}} \\ \hline 
\endfirsthead

\multicolumn{3}{c}%
{{\tablename\ \thetable{}---continued}} \\
\hline \multicolumn{1}{|c|}{\textbf{parameter}} & 
       \multicolumn{1}{ c|}{\textbf{description}} & 
       \multicolumn{1}{ c|}{\textbf{default value}} \\ \hline 
\endhead

\multicolumn{3}{|r|}{{\em continued on next page}} \\ \hline
\endfoot

\hline 
\endlastfoot


\rowcolor{tableShade}
\verb=  atol_enuc  = &    &  1.d-6 \\
\verb=  atol_spec  = &    &  1.d-12 \\
\rowcolor{tableShade}
\verb=  atol_temp  = &    &  1.d-6 \\
\verb=  burner_verbose  = &   Should we print out diagnostic output after the solve?  &  .false. \\
\rowcolor{tableShade}
\verb=  burning_mode  = &   Integration mode: if 0, a hydrostatic burn (temperature and density remain constant), and if 1, a self-heating burn (temperature/energy evolve with the burning). If 2, a hybrid approach presented by Raskin et al. (2010): do hydrostatic everywhere, but if the hydrostatic burn gives us a negative energy change, redo the burn in self-heating mode. If 3, do normal self-heating, but limit all values of the RHS by the same factor $f$ such that $\dot{e} = fe / t_s$, where enuc is the energy injection rate, $e$ is the internal energy of the zone, and $t_s$ is the sound crossing time.  &  1 \\
\verb=  burning_mode_factor  = &   If we're using burning\_mode == 3, this is the factor $f$ to use.  &  1.d-1 \\
\rowcolor{tableShade}
\verb=  call_eos_in_rhs  = &   Do we call the EOS each time we enter the EOS?  This is expensive, but more accurate.  Otherwise, we instead call the EOS at the start of the integration and freeze the thermodynamics throughout the RHS evalulation.  This only affects the temperature integration (which is the input to the rate evaluation). In particular, since we calculate the composition factors either way, this determines whether we're updating the thermodynamic derivatives and other quantities (cp and cv) as we go.  &  .false. \\
\verb=  centered_diff_jac  = &   one-sided numerical jacobian (.False.) or centered-difference Jacobian (.true.).  Note: the centered-difference requires twice as many RHS calls  &  .false. \\
\rowcolor{tableShade}
\verb=  dT_crit  = &   If we want to call the EOS in general, but don't want to overdo it, we can set a fraction dT\_crit such that we only do the EOS call if the temperature has changed by a relative fraction $>$ dT\_crit. If we use this option, we will do a linear fit to c\_v and c\_p in between EOS calls. This will work regardless of call\_eos\_in\_rhs.  &  1.0d20 \\
\verb=  integrate_energy  = &   Allow the energy integration to be disabled by setting the RHS to zero.  &  .true. \\
\rowcolor{tableShade}
\verb=  integrate_molar_fraction  = &   By default we integrate molar fraction Y. Disable this to integrate mass fraction X.  &  .true. \\
\verb=  integrate_temperature  = &   Allow the temperature integration to be disabled by setting the RHS to zero.  &  .true. \\
\rowcolor{tableShade}
\verb=  jacobian  = &   Whether to use an analytical or numerical Jacobian. 1 == Analytical 2 == Numerical  &  1 \\
\verb=  renormalize_abundances  = &   Whether to renormalize the mass fractions at each step in the evolution so that they sum to unity.  &  .false. \\
\rowcolor{tableShade}
\verb=  retry_burn  = &   If we fail to find a solution consistent with the tolerances, do we want to try again with a looser tolerance?  &  .false. \\
\verb=  retry_burn_factor  = &   If we do retry a burn, by what factor should we loosen the tolerance?  &  1.25d0 \\
\rowcolor{tableShade}
\verb=  retry_burn_max_change  = &   What is the maximum factor we can increase the original tolerances by?  &  1.0d2 \\
\verb=  rtol_enuc  = &    &  1.d-6 \\
\rowcolor{tableShade}
\verb=  rtol_spec  = &   Tolerances for the solver (relative and absolute), for the species, temperature, and energy equations.  &  1.d-12 \\
\verb=  rtol_temp  = &    &  1.d-6 \\


\end{longtable}
\end{center}

} % ends \small


{\small

\renewcommand{\arraystretch}{1.5}
%
\begin{center}
\begin{longtable}{|l|p{5.25in}|l|}
\caption[multigamma parameters.]{multigamma parameters.} \label{table: multigamma runtime} \\
%
\hline \multicolumn{1}{|c|}{\textbf{parameter}} & 
       \multicolumn{1}{ c|}{\textbf{description}} & 
       \multicolumn{1}{ c|}{\textbf{default value}} \\ \hline 
\endfirsthead

\multicolumn{3}{c}%
{{\tablename\ \thetable{}---continued}} \\
\hline \multicolumn{1}{|c|}{\textbf{parameter}} & 
       \multicolumn{1}{ c|}{\textbf{description}} & 
       \multicolumn{1}{ c|}{\textbf{default value}} \\ \hline 
\endhead

\multicolumn{3}{|r|}{{\em continued on next page}} \\ \hline
\endfoot

\hline 
\endlastfoot


\rowcolor{tableShade}
\verb=  species_a_gamma  = &    &  1.4 \\
\verb=  species_a_name  = &    &  "" \\
\rowcolor{tableShade}
\verb=  species_b_gamma  = &    &  1.4 \\
\verb=  species_b_name  = &    &  "" \\
\rowcolor{tableShade}
\verb=  species_c_gamma  = &    &  1.4 \\
\verb=  species_c_name  = &    &  "" \\


\end{longtable}
\end{center}

} % ends \small


{\small

\renewcommand{\arraystretch}{1.5}
%
\begin{center}
\begin{longtable}{|l|p{5.25in}|l|}
\caption[powerlaw parameters.]{powerlaw parameters.} \label{table: powerlaw runtime} \\
%
\hline \multicolumn{1}{|c|}{\textbf{parameter}} & 
       \multicolumn{1}{ c|}{\textbf{description}} & 
       \multicolumn{1}{ c|}{\textbf{default value}} \\ \hline 
\endfirsthead

\multicolumn{3}{c}%
{{\tablename\ \thetable{}---continued}} \\
\hline \multicolumn{1}{|c|}{\textbf{parameter}} & 
       \multicolumn{1}{ c|}{\textbf{description}} & 
       \multicolumn{1}{ c|}{\textbf{default value}} \\ \hline 
\endhead

\multicolumn{3}{|r|}{{\em continued on next page}} \\ \hline
\endfoot

\hline 
\endlastfoot


\rowcolor{tableShade}
\verb=  T_burn_ref  = &   reaction thresholds (for the power law)  &  1.0d0 \\
\verb=  burning_mode  = &   override the default burning mode with a higher priority  &  0 \\
\rowcolor{tableShade}
\verb=  f_act  = &    &  1.0d0 \\
\verb=  jacobian  = &   override the default Jacobian mode with a higher priority  &  2 \\
\rowcolor{tableShade}
\verb=  nu  = &   exponent for the temperature  &  4.d0 \\
\verb=  rho_burn_ref  = &    &  1.0d0 \\
\rowcolor{tableShade}
\verb=  rtilde  = &   the coefficient for the reaction rate  &  1.d0 \\
\verb=  specific_q_burn  = &   reaction specific q-value (in erg/g)  &  10.d0 \\


\end{longtable}
\end{center}

} % ends \small


{\small

\renewcommand{\arraystretch}{1.5}
%
\begin{center}
\begin{longtable}{|l|p{5.25in}|l|}
\caption[rprox parameters.]{rprox parameters.} \label{table: rprox runtime} \\
%
\hline \multicolumn{1}{|c|}{\textbf{parameter}} & 
       \multicolumn{1}{ c|}{\textbf{description}} & 
       \multicolumn{1}{ c|}{\textbf{default value}} \\ \hline 
\endfirsthead

\multicolumn{3}{c}%
{{\tablename\ \thetable{}---continued}} \\
\hline \multicolumn{1}{|c|}{\textbf{parameter}} & 
       \multicolumn{1}{ c|}{\textbf{description}} & 
       \multicolumn{1}{ c|}{\textbf{default value}} \\ \hline 
\endhead

\multicolumn{3}{|r|}{{\em continued on next page}} \\ \hline
\endfoot

\hline 
\endlastfoot


\rowcolor{tableShade}
\verb=  atol_enuc  = &    &  1.0e-8 \\
\verb=  atol_spec  = &   override the default tolerances for backwards compatibility  &  1.0e-11 \\
\rowcolor{tableShade}
\verb=  atol_temp  = &    &  1.0e-8 \\
\verb=  do_constant_volume_burn  = &   we typically run this network for constant-pressure burns  &  .false. \\
\rowcolor{tableShade}
\verb=  jacobian  = &   override so that the default is an analytical Jacobian  &  1 \\
\verb=  rtol_enuc  = &    &  1.0e-8 \\
\rowcolor{tableShade}
\verb=  rtol_spec  = &    &  1.0e-12 \\
\verb=  rtol_temp  = &    &  1.0e-8 \\


\end{longtable}
\end{center}

} % ends \small


{\small

\renewcommand{\arraystretch}{1.5}
%
\begin{center}
\begin{longtable}{|l|p{5.25in}|l|}
\caption[test\_eos parameters.]{test\_eos parameters.} \label{table: test_eos runtime} \\
%
\hline \multicolumn{1}{|c|}{\textbf{parameter}} & 
       \multicolumn{1}{ c|}{\textbf{description}} & 
       \multicolumn{1}{ c|}{\textbf{default value}} \\ \hline 
\endfirsthead

\multicolumn{3}{c}%
{{\tablename\ \thetable{}---continued}} \\
\hline \multicolumn{1}{|c|}{\textbf{parameter}} & 
       \multicolumn{1}{ c|}{\textbf{description}} & 
       \multicolumn{1}{ c|}{\textbf{default value}} \\ \hline 
\endhead

\multicolumn{3}{|r|}{{\em continued on next page}} \\ \hline
\endfoot

\hline 
\endlastfoot


\rowcolor{tableShade}
\verb=  metalicity_max  = &    &  0.1d0 \\
\verb=  small_dens  = &    &  1.e-4 \\
\rowcolor{tableShade}
\verb=  small_temp  = &    &  1.e4 \\
\verb=  test_set  = &    &  "gr0\_3d" \\
\rowcolor{tableShade}
\verb=  xin_file  = &    &  "uniform" \\


\end{longtable}
\end{center}

} % ends \small


{\small

\renewcommand{\arraystretch}{1.5}
%
\begin{center}
\begin{longtable}{|l|p{5.25in}|l|}
\caption[test\_react parameters.]{test\_react parameters.} \label{table: test_react runtime} \\
%
\hline \multicolumn{1}{|c|}{\textbf{parameter}} & 
       \multicolumn{1}{ c|}{\textbf{description}} & 
       \multicolumn{1}{ c|}{\textbf{default value}} \\ \hline 
\endfirsthead

\multicolumn{3}{c}%
{{\tablename\ \thetable{}---continued}} \\
\hline \multicolumn{1}{|c|}{\textbf{parameter}} & 
       \multicolumn{1}{ c|}{\textbf{description}} & 
       \multicolumn{1}{ c|}{\textbf{default value}} \\ \hline 
\endhead

\multicolumn{3}{|r|}{{\em continued on next page}} \\ \hline
\endfoot

\hline 
\endlastfoot


\rowcolor{tableShade}
\verb=  dt  = &    &  0.1d0 \\
\verb=  run_prefix  = &    &  "" \\
\rowcolor{tableShade}
\verb=  small_dens  = &    &  1.e5 \\
\verb=  small_temp  = &    &  1.e5 \\
\rowcolor{tableShade}
\verb=  test_set  = &    &  "gr0\_3d" \\
\verb=  xin_file  = &    &  "uniform" \\


\end{longtable}
\end{center}

} % ends \small


{\small

\renewcommand{\arraystretch}{1.5}
%
\begin{center}
\begin{longtable}{|l|p{5.25in}|l|}
\caption[triple\_alpha\_plus\_cago parameters.]{triple\_alpha\_plus\_cago parameters.} \label{table: triple_alpha_plus_cago runtime} \\
%
\hline \multicolumn{1}{|c|}{\textbf{parameter}} & 
       \multicolumn{1}{ c|}{\textbf{description}} & 
       \multicolumn{1}{ c|}{\textbf{default value}} \\ \hline 
\endfirsthead

\multicolumn{3}{c}%
{{\tablename\ \thetable{}---continued}} \\
\hline \multicolumn{1}{|c|}{\textbf{parameter}} & 
       \multicolumn{1}{ c|}{\textbf{description}} & 
       \multicolumn{1}{ c|}{\textbf{default value}} \\ \hline 
\endhead

\multicolumn{3}{|r|}{{\em continued on next page}} \\ \hline
\endfoot

\hline 
\endlastfoot


\rowcolor{tableShade}
\verb=  atol_enuc  = &    &  1.0e-8 \\
\verb=  atol_spec  = &   override the default tolerances for backwards compatibility  &  1.0e-12 \\
\rowcolor{tableShade}
\verb=  atol_temp  = &    &  1.0e-8 \\
\verb=  do_constant_volume_burn  = &   we typically run this network for constant-pressure burns  &  .false. \\
\rowcolor{tableShade}
\verb=  jacobian  = &   override so that the default is an analytical Jacobian  &  1 \\
\verb=  rtol_enuc  = &    &  1.0e-6 \\
\rowcolor{tableShade}
\verb=  rtol_spec  = &    &  1.0e-12 \\
\verb=  rtol_temp  = &    &  1.0e-6 \\


\end{longtable}
\end{center}

} % ends \small


{\small

\renewcommand{\arraystretch}{1.5}
%
\begin{center}
\begin{longtable}{|l|p{5.25in}|l|}
\caption[xrb\_simple parameters.]{xrb\_simple parameters.} \label{table: xrb_simple runtime} \\
%
\hline \multicolumn{1}{|c|}{\textbf{parameter}} & 
       \multicolumn{1}{ c|}{\textbf{description}} & 
       \multicolumn{1}{ c|}{\textbf{default value}} \\ \hline 
\endfirsthead

\multicolumn{3}{c}%
{{\tablename\ \thetable{}---continued}} \\
\hline \multicolumn{1}{|c|}{\textbf{parameter}} & 
       \multicolumn{1}{ c|}{\textbf{description}} & 
       \multicolumn{1}{ c|}{\textbf{default value}} \\ \hline 
\endhead

\multicolumn{3}{|r|}{{\em continued on next page}} \\ \hline
\endfoot

\hline 
\endlastfoot


\rowcolor{tableShade}
\verb=  atol_enuc  = &    &  1.0e-8 \\
\verb=  atol_spec  = &   override the default tolerances for backwards compatibility  &  1.0e-11 \\
\rowcolor{tableShade}
\verb=  atol_temp  = &    &  1.0e-8 \\
\verb=  do_constant_volume_burn  = &   we typically run this network for constant-pressure burns  &  .false. \\
\rowcolor{tableShade}
\verb=  jacobian  = &   override so that the default is a numerical Jacobian; we don't yet have an analytical Jacobian  &  2 \\
\verb=  rtol_enuc  = &    &  1.0e-8 \\
\rowcolor{tableShade}
\verb=  rtol_spec  = &    &  1.0e-12 \\
\verb=  rtol_temp  = &    &  1.0e-8 \\


\end{longtable}
\end{center}

} % ends \small


\end{landscape}

%


