
\label{ch:parameters}

\begin{landscape}


{\small

\renewcommand{\arraystretch}{1.5}
%
\begin{center}
\begin{longtable}{|l|p{5.25in}|l|}
\caption[BS parameters.]{BS parameters.} \label{table: BS runtime} \\
%
\hline \multicolumn{1}{|c|}{\textbf{parameter}} &
       \multicolumn{1}{ c|}{\textbf{description}} &
       \multicolumn{1}{ c|}{\textbf{default value}} \\ \hline
\endfirsthead

\multicolumn{3}{c}%
{{\tablename\ \thetable{}---continued}} \\
\hline \multicolumn{1}{|c|}{\textbf{parameter}} &
       \multicolumn{1}{ c|}{\textbf{description}} &
       \multicolumn{1}{ c|}{\textbf{default value}} \\ \hline
\endhead

\multicolumn{3}{|r|}{{\em continued on next page}} \\ \hline
\endfoot

\hline
\endlastfoot


\rowcolor{tableShade}
\verb= ode_max_steps = &  Maximum number of steps to use in the ODE integration & 10000 \\
\verb= ode_method = &  use an implementation of the Bulirsch-Stoer semi-implicit extrapolation method (1) or a Rosenbrock method (2) & 1 \\
\rowcolor{tableShade}
\verb= ode_scale_floor = &  Floor to use for the ODE scaling vector & 1.d-6 \\
\verb= safety_factor = &  when constructing the intermediate steps in the stiff ODE integration by how much do we allow the state variables to change over a dt before giving up on the step and retrying with a smaller step? & 1.d9 \\
\rowcolor{tableShade}
\verb= scaling_method = &  Which choice to use for the ODE scaling 1: $|y| + |dy/dt|$; 2: $\max(|y|, K)$ with $K =$ constant & 2 \\
\verb= use_timestep_estimator = &  use the VODE algorithm's initial timestep estimator? & .false. \\


\end{longtable}
\end{center}

} % ends \small


{\small

\renewcommand{\arraystretch}{1.5}
%
\begin{center}
\begin{longtable}{|l|p{5.25in}|l|}
\caption[VBDF parameters.]{VBDF parameters.} \label{table: VBDF runtime} \\
%
\hline \multicolumn{1}{|c|}{\textbf{parameter}} &
       \multicolumn{1}{ c|}{\textbf{description}} &
       \multicolumn{1}{ c|}{\textbf{default value}} \\ \hline
\endfirsthead

\multicolumn{3}{c}%
{{\tablename\ \thetable{}---continued}} \\
\hline \multicolumn{1}{|c|}{\textbf{parameter}} &
       \multicolumn{1}{ c|}{\textbf{description}} &
       \multicolumn{1}{ c|}{\textbf{default value}} \\ \hline
\endhead

\multicolumn{3}{|r|}{{\em continued on next page}} \\ \hline
\endfoot

\hline
\endlastfoot


\rowcolor{tableShade}
\verb= dt_min = &  minimum allowable timestep & 1.d-24 \\
\verb= jac_age = &  number of times we can use the Jacobian before rebuilding & 50 \\
\rowcolor{tableShade}
\verb= p_age = &  number of times we use the same Newton iteration matrix before rebuilding & 20 \\
\verb= reuse_jac = &  reuse the Jacobian? & .false. \\


\end{longtable}
\end{center}

} % ends \small


{\small

\renewcommand{\arraystretch}{1.5}
%
\begin{center}
\begin{longtable}{|l|p{5.25in}|l|}
\caption[breakout parameters.]{breakout parameters.} \label{table: breakout runtime} \\
%
\hline \multicolumn{1}{|c|}{\textbf{parameter}} &
       \multicolumn{1}{ c|}{\textbf{description}} &
       \multicolumn{1}{ c|}{\textbf{default value}} \\ \hline
\endfirsthead

\multicolumn{3}{c}%
{{\tablename\ \thetable{}---continued}} \\
\hline \multicolumn{1}{|c|}{\textbf{parameter}} &
       \multicolumn{1}{ c|}{\textbf{description}} &
       \multicolumn{1}{ c|}{\textbf{default value}} \\ \hline
\endhead

\multicolumn{3}{|r|}{{\em continued on next page}} \\ \hline
\endfoot

\hline
\endlastfoot


\rowcolor{tableShade}
\verb= eos_gamma = &  & 0.d0 \\


\end{longtable}
\end{center}

} % ends \small


{\small

\renewcommand{\arraystretch}{1.5}
%
\begin{center}
\begin{longtable}{|l|p{5.25in}|l|}
\caption[burn\_cell parameters.]{burn\_cell parameters.} \label{table: burn_cell runtime} \\
%
\hline \multicolumn{1}{|c|}{\textbf{parameter}} &
       \multicolumn{1}{ c|}{\textbf{description}} &
       \multicolumn{1}{ c|}{\textbf{default value}} \\ \hline
\endfirsthead

\multicolumn{3}{c}%
{{\tablename\ \thetable{}---continued}} \\
\hline \multicolumn{1}{|c|}{\textbf{parameter}} &
       \multicolumn{1}{ c|}{\textbf{description}} &
       \multicolumn{1}{ c|}{\textbf{default value}} \\ \hline
\endhead

\multicolumn{3}{|r|}{{\em continued on next page}} \\ \hline
\endfoot

\hline
\endlastfoot


\rowcolor{tableShade}
\verb= run_prefix = &  & "" \\
\verb= small_dens = &  & 1.e5 \\
\rowcolor{tableShade}
\verb= small_temp = &  & 1.e5 \\


\end{longtable}
\end{center}

} % ends \small


{\small

\renewcommand{\arraystretch}{1.5}
%
\begin{center}
\begin{longtable}{|l|p{5.25in}|l|}
\caption[cj\_detonation parameters.]{cj\_detonation parameters.} \label{table: cj_detonation runtime} \\
%
\hline \multicolumn{1}{|c|}{\textbf{parameter}} &
       \multicolumn{1}{ c|}{\textbf{description}} &
       \multicolumn{1}{ c|}{\textbf{default value}} \\ \hline
\endfirsthead

\multicolumn{3}{c}%
{{\tablename\ \thetable{}---continued}} \\
\hline \multicolumn{1}{|c|}{\textbf{parameter}} &
       \multicolumn{1}{ c|}{\textbf{description}} &
       \multicolumn{1}{ c|}{\textbf{default value}} \\ \hline
\endhead

\multicolumn{3}{|r|}{{\em continued on next page}} \\ \hline
\endfoot

\hline
\endlastfoot


\rowcolor{tableShade}
\verb= smallx = &  & 1.e-10 \\


\end{longtable}
\end{center}

} % ends \small


{\small

\renewcommand{\arraystretch}{1.5}
%
\begin{center}
\begin{longtable}{|l|p{5.25in}|l|}
\caption[gamma\_law\_general parameters.]{gamma\_law\_general parameters.} \label{table: gamma_law_general runtime} \\
%
\hline \multicolumn{1}{|c|}{\textbf{parameter}} &
       \multicolumn{1}{ c|}{\textbf{description}} &
       \multicolumn{1}{ c|}{\textbf{default value}} \\ \hline
\endfirsthead

\multicolumn{3}{c}%
{{\tablename\ \thetable{}---continued}} \\
\hline \multicolumn{1}{|c|}{\textbf{parameter}} &
       \multicolumn{1}{ c|}{\textbf{description}} &
       \multicolumn{1}{ c|}{\textbf{default value}} \\ \hline
\endhead

\multicolumn{3}{|r|}{{\em continued on next page}} \\ \hline
\endfoot

\hline
\endlastfoot


\rowcolor{tableShade}
\verb= eos_assume_neutral = &  & .true. \\
\verb= eos_gamma = &  & 5.d0/3.d0 \\


\end{longtable}
\end{center}

} % ends \small


{\small

\renewcommand{\arraystretch}{1.5}
%
\begin{center}
\begin{longtable}{|l|p{5.25in}|l|}
\caption[helmholtz parameters.]{helmholtz parameters.} \label{table: helmholtz runtime} \\
%
\hline \multicolumn{1}{|c|}{\textbf{parameter}} &
       \multicolumn{1}{ c|}{\textbf{description}} &
       \multicolumn{1}{ c|}{\textbf{default value}} \\ \hline
\endfirsthead

\multicolumn{3}{c}%
{{\tablename\ \thetable{}---continued}} \\
\hline \multicolumn{1}{|c|}{\textbf{parameter}} &
       \multicolumn{1}{ c|}{\textbf{description}} &
       \multicolumn{1}{ c|}{\textbf{default value}} \\ \hline
\endhead

\multicolumn{3}{|r|}{{\em continued on next page}} \\ \hline
\endfoot

\hline
\endlastfoot


\rowcolor{tableShade}
\verb= eos_input_is_constant = &  Force the EOS output quantities to match input & .false. \\
\verb= use_eos_coulomb = &  use the Coulomb corrections & .true. \\


\end{longtable}
\end{center}

} % ends \small


{\small

\renewcommand{\arraystretch}{1.5}
%
\begin{center}
\begin{longtable}{|l|p{5.25in}|l|}
\caption[integration parameters.]{integration parameters.} \label{table: integration runtime} \\
%
\hline \multicolumn{1}{|c|}{\textbf{parameter}} &
       \multicolumn{1}{ c|}{\textbf{description}} &
       \multicolumn{1}{ c|}{\textbf{default value}} \\ \hline
\endfirsthead

\multicolumn{3}{c}%
{{\tablename\ \thetable{}---continued}} \\
\hline \multicolumn{1}{|c|}{\textbf{parameter}} &
       \multicolumn{1}{ c|}{\textbf{description}} &
       \multicolumn{1}{ c|}{\textbf{default value}} \\ \hline
\endhead

\multicolumn{3}{|r|}{{\em continued on next page}} \\ \hline
\endfoot

\hline
\endlastfoot


\rowcolor{tableShade}
\verb= MAX_TEMP = &  The maximum temperature for reactions in the integration. & 1.0d11 \\
\verb= SMALL_X_SAFE = &  The absolute cutoff for species -- note that this might be larger than {\tt small\_x}, but the issue is that we need to prevent underflow issues and keep mass fractions positive in the integrator.  You may have to increase the floor to, e.g. {\tt 1.d-20} if your rates are large. & 1.0d-30 \\
\rowcolor{tableShade}
\verb= atol_enuc = &  & 1.d-6 \\
\verb= atol_spec = &  & 1.d-12 \\
\rowcolor{tableShade}
\verb= atol_temp = &  & 1.d-6 \\
\verb= burner_verbose = &  Should we print out diagnostic output after the solve? & .false. \\
\rowcolor{tableShade}
\verb= burning_mode = &  Integration mode: if 0, a hydrostatic burn (temperature and density remain constant), and if 1, a self-heating burn (temperature/energy evolve with the burning). If 2, a hybrid approach presented by Raskin et al. (2010): do hydrostatic everywhere, but if the hydrostatic burn gives us a negative energy change, redo the burn in self-heating mode.  If 3, do normal self-heating, but limit all values of the RHS by the same factor $L$ such that $\dot{e} = f_s e / t_s$, where $\dot{e}$ is the energy injection rate, $e$ is the internal energy of the zone, $t_s$ is the sound crossing time, and $f_s$ is a safety factor. $L$ is computed as min(1, $f_s (e / \dot{e}) / t_s$). & 1 \\
\verb= burning_mode_factor = &  If we're using burning\_mode == 3, this is the safety factor $f_s$ to use. & 1.d-1 \\
\rowcolor{tableShade}
\verb= call_eos_in_rhs = &  Do we call the EOS each time we enter the EOS?  This is expensive, but more accurate.  Otherwise, we instead call the EOS at the start of the integration and freeze the thermodynamics throughout the RHS evalulation.  This only affects the temperature integration (which is the input to the rate evaluation). In particular, since we calculate the composition factors either way, this determines whether we're updating the thermodynamic derivatives and other quantities (cp and cv) as we go. & .false. \\
\verb= centered_diff_jac = &  one-sided numerical jacobian (.False.) or centered-difference Jacobian (.true.).  Note: the centered-difference requires twice as many RHS calls & .false. \\
\rowcolor{tableShade}
\verb= dT_crit = &  If we want to call the EOS in general, but don't want to overdo it, we can set a fraction dT\_crit such that we only do the EOS call if the temperature has changed by a relative fraction $>$ dT\_crit. If we use this option, we will do a linear fit to c\_v and c\_p in between EOS calls. This will work regardless of call\_eos\_in\_rhs. & 1.0d20 \\
\verb= do_constant_volume_burn = &  When evolving the temperature, should we assume a constant pressure (default) or a constant volume (do\_constant\_volume\_burn = T)? & .false. \\
\rowcolor{tableShade}
\verb= integrate_energy = &  Allow the energy integration to be disabled by setting the RHS to zero. & .true. \\
\verb= integrate_temperature = &  Allow the temperature integration to be disabled by setting the RHS to zero. & .true. \\
\rowcolor{tableShade}
\verb= jacobian = &  Whether to use an analytical or numerical Jacobian. 1 == Analytical 2 == Numerical & 1 \\
\verb= react_boost = &  boost the reaction rates by a factor > 1 & -1.d0 \\
\rowcolor{tableShade}
\verb= renormalize_abundances = &  Whether to renormalize the mass fractions at each step in the evolution so that they sum to unity. & .false. \\
\verb= retry_burn = &  If we fail to find a solution consistent with the tolerances, do we want to try again with a looser tolerance? & .false. \\
\rowcolor{tableShade}
\verb= retry_burn_factor = &  If we do retry a burn, by what factor should we loosen the tolerance? & 1.25d0 \\
\verb= retry_burn_max_change = &  What is the maximum factor we can increase the original tolerances by? & 1.0d2 \\
\rowcolor{tableShade}
\verb= rtol_enuc = &  & 1.d-6 \\
\verb= rtol_spec = &  Tolerances for the solver (relative and absolute), for the species, temperature, and energy equations. & 1.d-12 \\
\rowcolor{tableShade}
\verb= rtol_temp = &  & 1.d-6 \\


\end{longtable}
\end{center}

} % ends \small


{\small

\renewcommand{\arraystretch}{1.5}
%
\begin{center}
\begin{longtable}{|l|p{5.25in}|l|}
\caption[kpp parameters.]{kpp parameters.} \label{table: kpp runtime} \\
%
\hline \multicolumn{1}{|c|}{\textbf{parameter}} &
       \multicolumn{1}{ c|}{\textbf{description}} &
       \multicolumn{1}{ c|}{\textbf{default value}} \\ \hline
\endfirsthead

\multicolumn{3}{c}%
{{\tablename\ \thetable{}---continued}} \\
\hline \multicolumn{1}{|c|}{\textbf{parameter}} &
       \multicolumn{1}{ c|}{\textbf{description}} &
       \multicolumn{1}{ c|}{\textbf{default value}} \\ \hline
\endhead

\multicolumn{3}{|r|}{{\em continued on next page}} \\ \hline
\endfoot

\hline
\endlastfoot


\rowcolor{tableShade}
\verb= A_burn = &  & 10.d0 \\


\end{longtable}
\end{center}

} % ends \small


{\small

\renewcommand{\arraystretch}{1.5}
%
\begin{center}
\begin{longtable}{|l|p{5.25in}|l|}
\caption[multigamma parameters.]{multigamma parameters.} \label{table: multigamma runtime} \\
%
\hline \multicolumn{1}{|c|}{\textbf{parameter}} &
       \multicolumn{1}{ c|}{\textbf{description}} &
       \multicolumn{1}{ c|}{\textbf{default value}} \\ \hline
\endfirsthead

\multicolumn{3}{c}%
{{\tablename\ \thetable{}---continued}} \\
\hline \multicolumn{1}{|c|}{\textbf{parameter}} &
       \multicolumn{1}{ c|}{\textbf{description}} &
       \multicolumn{1}{ c|}{\textbf{default value}} \\ \hline
\endhead

\multicolumn{3}{|r|}{{\em continued on next page}} \\ \hline
\endfoot

\hline
\endlastfoot


\rowcolor{tableShade}
\verb= eos_gamma_default = &  & 1.4 \\
\verb= species_a_gamma = &  & 1.4 \\
\rowcolor{tableShade}
\verb= species_a_name = &  & "" \\
\verb= species_b_gamma = &  & 1.4 \\
\rowcolor{tableShade}
\verb= species_b_name = &  & "" \\
\verb= species_c_gamma = &  & 1.4 \\
\rowcolor{tableShade}
\verb= species_c_name = &  & "" \\


\end{longtable}
\end{center}

} % ends \small


{\small

\renewcommand{\arraystretch}{1.5}
%
\begin{center}
\begin{longtable}{|l|p{5.25in}|l|}
\caption[networks parameters.]{networks parameters.} \label{table: networks runtime} \\
%
\hline \multicolumn{1}{|c|}{\textbf{parameter}} &
       \multicolumn{1}{ c|}{\textbf{description}} &
       \multicolumn{1}{ c|}{\textbf{default value}} \\ \hline
\endfirsthead

\multicolumn{3}{c}%
{{\tablename\ \thetable{}---continued}} \\
\hline \multicolumn{1}{|c|}{\textbf{parameter}} &
       \multicolumn{1}{ c|}{\textbf{description}} &
       \multicolumn{1}{ c|}{\textbf{default value}} \\ \hline
\endhead

\multicolumn{3}{|r|}{{\em continued on next page}} \\ \hline
\endfoot

\hline
\endlastfoot


\rowcolor{tableShade}
\verb= small_x = &  cutoff for species mass fractions & 1.d-30 \\
\verb= use_c12ag_deboer17 = &  Should we use Deboer + 2017 rate for c12(a,g)o16? & .false. \\
\rowcolor{tableShade}
\verb= use_tables = &  Should we use rate tables if they are present in the network? & .false. \\


\end{longtable}
\end{center}

} % ends \small


{\small

\renewcommand{\arraystretch}{1.5}
%
\begin{center}
\begin{longtable}{|l|p{5.25in}|l|}
\caption[polytrope parameters.]{polytrope parameters.} \label{table: polytrope runtime} \\
%
\hline \multicolumn{1}{|c|}{\textbf{parameter}} &
       \multicolumn{1}{ c|}{\textbf{description}} &
       \multicolumn{1}{ c|}{\textbf{default value}} \\ \hline
\endfirsthead

\multicolumn{3}{c}%
{{\tablename\ \thetable{}---continued}} \\
\hline \multicolumn{1}{|c|}{\textbf{parameter}} &
       \multicolumn{1}{ c|}{\textbf{description}} &
       \multicolumn{1}{ c|}{\textbf{default value}} \\ \hline
\endhead

\multicolumn{3}{|r|}{{\em continued on next page}} \\ \hline
\endfoot

\hline
\endlastfoot


\rowcolor{tableShade}
\verb= polytrope_K = &  & 0.0d0 \\
\verb= polytrope_gamma = &  & 0.0d0 \\
\rowcolor{tableShade}
\verb= polytrope_mu_e = &  & 2.0d0 \\
\verb= polytrope_type = &  & 0 \\


\end{longtable}
\end{center}

} % ends \small


{\small

\renewcommand{\arraystretch}{1.5}
%
\begin{center}
\begin{longtable}{|l|p{5.25in}|l|}
\caption[powerlaw parameters.]{powerlaw parameters.} \label{table: powerlaw runtime} \\
%
\hline \multicolumn{1}{|c|}{\textbf{parameter}} &
       \multicolumn{1}{ c|}{\textbf{description}} &
       \multicolumn{1}{ c|}{\textbf{default value}} \\ \hline
\endfirsthead

\multicolumn{3}{c}%
{{\tablename\ \thetable{}---continued}} \\
\hline \multicolumn{1}{|c|}{\textbf{parameter}} &
       \multicolumn{1}{ c|}{\textbf{description}} &
       \multicolumn{1}{ c|}{\textbf{default value}} \\ \hline
\endhead

\multicolumn{3}{|r|}{{\em continued on next page}} \\ \hline
\endfoot

\hline
\endlastfoot


\rowcolor{tableShade}
\verb= T_burn_ref = &  reaction thresholds (for the power law) & 1.0d0 \\
\verb= burning_mode = &  override the default burning mode with a higher priority & 0 \\
\rowcolor{tableShade}
\verb= f_act = &  & 1.0d0 \\
\verb= jacobian = &  override the default Jacobian mode with a higher priority & 2 \\
\rowcolor{tableShade}
\verb= nu = &  exponent for the temperature & 4.d0 \\
\verb= rho_burn_ref = &  & 1.0d0 \\
\rowcolor{tableShade}
\verb= rtilde = &  the coefficient for the reaction rate & 1.d0 \\
\verb= specific_q_burn = &  reaction specific q-value (in erg/g) & 10.d0 \\


\end{longtable}
\end{center}

} % ends \small


{\small

\renewcommand{\arraystretch}{1.5}
%
\begin{center}
\begin{longtable}{|l|p{5.25in}|l|}
\caption[rprox parameters.]{rprox parameters.} \label{table: rprox runtime} \\
%
\hline \multicolumn{1}{|c|}{\textbf{parameter}} &
       \multicolumn{1}{ c|}{\textbf{description}} &
       \multicolumn{1}{ c|}{\textbf{default value}} \\ \hline
\endfirsthead

\multicolumn{3}{c}%
{{\tablename\ \thetable{}---continued}} \\
\hline \multicolumn{1}{|c|}{\textbf{parameter}} &
       \multicolumn{1}{ c|}{\textbf{description}} &
       \multicolumn{1}{ c|}{\textbf{default value}} \\ \hline
\endhead

\multicolumn{3}{|r|}{{\em continued on next page}} \\ \hline
\endfoot

\hline
\endlastfoot


\rowcolor{tableShade}
\verb= atol_enuc = &  & 1.0e-8 \\
\verb= atol_spec = &  override the default tolerances for backwards compatibility & 1.0e-11 \\
\rowcolor{tableShade}
\verb= atol_temp = &  & 1.0e-8 \\
\verb= burning_mode = &  override the default burning mode with a higher priority & 1 \\
\rowcolor{tableShade}
\verb= do_constant_volume_burn = &  we typically run this network for constant-pressure burns & .false. \\
\verb= jacobian = &  override so that the default is an analytical Jacobian & 1 \\
\rowcolor{tableShade}
\verb= rtol_enuc = &  & 1.0e-8 \\
\verb= rtol_spec = &  & 1.0e-12 \\
\rowcolor{tableShade}
\verb= rtol_temp = &  & 1.0e-8 \\


\end{longtable}
\end{center}

} % ends \small


{\small

\renewcommand{\arraystretch}{1.5}
%
\begin{center}
\begin{longtable}{|l|p{5.25in}|l|}
\caption[stellarcollapse parameters.]{stellarcollapse parameters.} \label{table: stellarcollapse runtime} \\
%
\hline \multicolumn{1}{|c|}{\textbf{parameter}} &
       \multicolumn{1}{ c|}{\textbf{description}} &
       \multicolumn{1}{ c|}{\textbf{default value}} \\ \hline
\endfirsthead

\multicolumn{3}{c}%
{{\tablename\ \thetable{}---continued}} \\
\hline \multicolumn{1}{|c|}{\textbf{parameter}} &
       \multicolumn{1}{ c|}{\textbf{description}} &
       \multicolumn{1}{ c|}{\textbf{default value}} \\ \hline
\endhead

\multicolumn{3}{|r|}{{\em continued on next page}} \\ \hline
\endfoot

\hline
\endlastfoot


\rowcolor{tableShade}
\verb= eos_file = &  name of the HDF5 file containing tabulated data & "" \\
\verb= use_energy_shift = &  & .false. \\


\end{longtable}
\end{center}

} % ends \small


{\small

\renewcommand{\arraystretch}{1.5}
%
\begin{center}
\begin{longtable}{|l|p{5.25in}|l|}
\caption[test\_eos parameters.]{test\_eos parameters.} \label{table: test_eos runtime} \\
%
\hline \multicolumn{1}{|c|}{\textbf{parameter}} &
       \multicolumn{1}{ c|}{\textbf{description}} &
       \multicolumn{1}{ c|}{\textbf{default value}} \\ \hline
\endfirsthead

\multicolumn{3}{c}%
{{\tablename\ \thetable{}---continued}} \\
\hline \multicolumn{1}{|c|}{\textbf{parameter}} &
       \multicolumn{1}{ c|}{\textbf{description}} &
       \multicolumn{1}{ c|}{\textbf{default value}} \\ \hline
\endhead

\multicolumn{3}{|r|}{{\em continued on next page}} \\ \hline
\endfoot

\hline
\endlastfoot


\rowcolor{tableShade}
\verb= dens_max = &  & 1.d9 \\
\verb= dens_min = &  & 1.d6 \\
\rowcolor{tableShade}
\verb= metalicity_max = &  & 0.1d0 \\
\verb= small_dens = &  & 1.e-4 \\
\rowcolor{tableShade}
\verb= small_temp = &  & 1.e4 \\
\verb= temp_max = &  & 1.d12 \\
\rowcolor{tableShade}
\verb= temp_min = &  & 1.d6 \\
\verb= test_set = &  & "gr0\_3d" \\
\rowcolor{tableShade}
\verb= xin_file = &  & "uniform" \\


\end{longtable}
\end{center}

} % ends \small


{\small

\renewcommand{\arraystretch}{1.5}
%
\begin{center}
\begin{longtable}{|l|p{5.25in}|l|}
\caption[test\_react parameters.]{test\_react parameters.} \label{table: test_react runtime} \\
%
\hline \multicolumn{1}{|c|}{\textbf{parameter}} &
       \multicolumn{1}{ c|}{\textbf{description}} &
       \multicolumn{1}{ c|}{\textbf{default value}} \\ \hline
\endfirsthead

\multicolumn{3}{c}%
{{\tablename\ \thetable{}---continued}} \\
\hline \multicolumn{1}{|c|}{\textbf{parameter}} &
       \multicolumn{1}{ c|}{\textbf{description}} &
       \multicolumn{1}{ c|}{\textbf{default value}} \\ \hline
\endhead

\multicolumn{3}{|r|}{{\em continued on next page}} \\ \hline
\endfoot

\hline
\endlastfoot


\rowcolor{tableShade}
\verb= dens_max = &  & 1.d9 \\
\verb= dens_min = &  & 1.d6 \\
\rowcolor{tableShade}
\verb= do_acc = &  & 1 \\
\verb= run_prefix = &  & "" \\
\rowcolor{tableShade}
\verb= small_dens = &  & 1.e5 \\
\verb= small_temp = &  & 1.e5 \\
\rowcolor{tableShade}
\verb= temp_max = &  & 1.d15 \\
\verb= temp_min = &  & 1.d6 \\
\rowcolor{tableShade}
\verb= test_set = &  & "gr0\_3d" \\
\verb= tmax = &  & 0.1d0 \\
\rowcolor{tableShade}
\verb= xin_file = &  & "uniform" \\


\end{longtable}
\end{center}

} % ends \small


{\small

\renewcommand{\arraystretch}{1.5}
%
\begin{center}
\begin{longtable}{|l|p{5.25in}|l|}
\caption[test\_sdc parameters.]{test\_sdc parameters.} \label{table: test_sdc runtime} \\
%
\hline \multicolumn{1}{|c|}{\textbf{parameter}} &
       \multicolumn{1}{ c|}{\textbf{description}} &
       \multicolumn{1}{ c|}{\textbf{default value}} \\ \hline
\endfirsthead

\multicolumn{3}{c}%
{{\tablename\ \thetable{}---continued}} \\
\hline \multicolumn{1}{|c|}{\textbf{parameter}} &
       \multicolumn{1}{ c|}{\textbf{description}} &
       \multicolumn{1}{ c|}{\textbf{default value}} \\ \hline
\endhead

\multicolumn{3}{|r|}{{\em continued on next page}} \\ \hline
\endfoot

\hline
\endlastfoot


\rowcolor{tableShade}
\verb= dens_max = &  & 1.d9 \\
\verb= dens_min = &  & 1.d6 \\
\rowcolor{tableShade}
\verb= do_acc = &  & 1 \\
\verb= run_prefix = &  & "" \\
\rowcolor{tableShade}
\verb= small_dens = &  & 1.e5 \\
\verb= small_temp = &  & 1.e5 \\
\rowcolor{tableShade}
\verb= temp_max = &  & 1.d15 \\
\verb= temp_min = &  & 1.d6 \\
\rowcolor{tableShade}
\verb= test_set = &  & "gr0\_3d" \\
\verb= tmax = &  & 0.1d0 \\
\rowcolor{tableShade}
\verb= xin_file = &  & "uniform" \\


\end{longtable}
\end{center}

} % ends \small


{\small

\renewcommand{\arraystretch}{1.5}
%
\begin{center}
\begin{longtable}{|l|p{5.25in}|l|}
\caption[triple\_alpha\_plus\_cago parameters.]{triple\_alpha\_plus\_cago parameters.} \label{table: triple_alpha_plus_cago runtime} \\
%
\hline \multicolumn{1}{|c|}{\textbf{parameter}} &
       \multicolumn{1}{ c|}{\textbf{description}} &
       \multicolumn{1}{ c|}{\textbf{default value}} \\ \hline
\endfirsthead

\multicolumn{3}{c}%
{{\tablename\ \thetable{}---continued}} \\
\hline \multicolumn{1}{|c|}{\textbf{parameter}} &
       \multicolumn{1}{ c|}{\textbf{description}} &
       \multicolumn{1}{ c|}{\textbf{default value}} \\ \hline
\endhead

\multicolumn{3}{|r|}{{\em continued on next page}} \\ \hline
\endfoot

\hline
\endlastfoot


\rowcolor{tableShade}
\verb= atol_enuc = &  & 1.0e-8 \\
\verb= atol_spec = &  override the default tolerances for backwards compatibility & 1.0e-12 \\
\rowcolor{tableShade}
\verb= atol_temp = &  & 1.0e-8 \\
\verb= burning_mode = &  override the default burning mode with a higher priority & 1 \\
\rowcolor{tableShade}
\verb= do_constant_volume_burn = &  we typically run this network for constant-pressure burns & .false. \\
\verb= jacobian = &  override so that the default is an analytical Jacobian & 1 \\
\rowcolor{tableShade}
\verb= rtol_enuc = &  & 1.0e-6 \\
\verb= rtol_spec = &  & 1.0e-12 \\
\rowcolor{tableShade}
\verb= rtol_temp = &  & 1.0e-6 \\


\end{longtable}
\end{center}

} % ends \small


{\small

\renewcommand{\arraystretch}{1.5}
%
\begin{center}
\begin{longtable}{|l|p{5.25in}|l|}
\caption[xrb\_simple parameters.]{xrb\_simple parameters.} \label{table: xrb_simple runtime} \\
%
\hline \multicolumn{1}{|c|}{\textbf{parameter}} &
       \multicolumn{1}{ c|}{\textbf{description}} &
       \multicolumn{1}{ c|}{\textbf{default value}} \\ \hline
\endfirsthead

\multicolumn{3}{c}%
{{\tablename\ \thetable{}---continued}} \\
\hline \multicolumn{1}{|c|}{\textbf{parameter}} &
       \multicolumn{1}{ c|}{\textbf{description}} &
       \multicolumn{1}{ c|}{\textbf{default value}} \\ \hline
\endhead

\multicolumn{3}{|r|}{{\em continued on next page}} \\ \hline
\endfoot

\hline
\endlastfoot


\rowcolor{tableShade}
\verb= atol_enuc = &  & 1.0e-8 \\
\verb= atol_spec = &  override the default tolerances for backwards compatibility & 1.0e-11 \\
\rowcolor{tableShade}
\verb= atol_temp = &  & 1.0e-8 \\
\verb= burning_mode = &  override the default burning mode with a higher priority & 1 \\
\rowcolor{tableShade}
\verb= do_constant_volume_burn = &  we typically run this network for constant-pressure burns & .false. \\
\verb= jacobian = &  override so that the default is a numerical Jacobian; we don't yet have an analytical Jacobian & 2 \\
\rowcolor{tableShade}
\verb= rtol_enuc = &  & 1.0e-8 \\
\verb= rtol_spec = &  & 1.0e-12 \\
\rowcolor{tableShade}
\verb= rtol_temp = &  & 1.0e-8 \\


\end{longtable}
\end{center}

} % ends \small


\end{landscape}

%


